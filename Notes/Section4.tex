%%%%%%%%%%%%%%%%%%%%%%%%%%%%%%%%%%%%%%%%%%%%%%%%%%%%%%%%%%%%%%%%%%%%%%%%%%%%%%%%%%%%%%%%%%%%%%%%%%%%%%%%%%%%%%%%%%%%%%%%%%%%%%%%%%%%%%%%%%%%%%%%%%%%%%%%%%%%%%%%%%%
% Written By Michael Brodskiy
% Class: Calculus III (MATH2321)
% Professor: A. Martsinkovsky
%%%%%%%%%%%%%%%%%%%%%%%%%%%%%%%%%%%%%%%%%%%%%%%%%%%%%%%%%%%%%%%%%%%%%%%%%%%%%%%%%%%%%%%%%%%%%%%%%%%%%%%%%%%%%%%%%%%%%%%%%%%%%%%%%%%%%%%%%%%%%%%%%%%%%%%%%%%%%%%%%%%

%%%%%%%%%%%%%%%%%%%%%%%%%%%%%%%%%%%%%%%%%%%%%%%%%%%%%%%%%%%%%%%%%%%%%%%%%%%%%%%%%%%%%%%%%%%%%%%%%%%%%%%%%%%%%%%%%%%%%%%%%%%%%%%%%%%%%%%%%%%%%%%%%%%%%%%%%%%%%%%%%%%
% Written By Michael Brodskiy
% Class: Calculus III (MATH2321)
% Professor: A. Martsinkovsky
%%%%%%%%%%%%%%%%%%%%%%%%%%%%%%%%%%%%%%%%%%%%%%%%%%%%%%%%%%%%%%%%%%%%%%%%%%%%%%%%%%%%%%%%%%%%%%%%%%%%%%%%%%%%%%%%%%%%%%%%%%%%%%%%%%%%%%%%%%%%%%%%%%%%%%%%%%%%%%%%%%%

\documentclass[12pt]{article} 
\usepackage{alphalph}
\usepackage[utf8]{inputenc}
\usepackage[russian,english]{babel}
\usepackage{titling}
\usepackage{amsmath}
\usepackage{graphicx}
\usepackage{enumitem}
\usepackage{amssymb}
\usepackage[super]{nth}
\usepackage{everysel}
\usepackage{ragged2e}
\usepackage{geometry}
\usepackage{multicol}
\usepackage{fancyhdr}
\usepackage{cancel}
\usepackage{siunitx}
\geometry{top=1.0in,bottom=1.0in,left=1.0in,right=1.0in}
\newcommand{\subtitle}[1]{%
  \posttitle{%
    \par\end{center}
    \begin{center}\large#1\end{center}
    \vskip0.5em}%

}
\usepackage{hyperref}
\hypersetup{
colorlinks=true,
linkcolor=blue,
filecolor=magenta,      
urlcolor=blue,
citecolor=blue,
}


\title{Section 4}
\date{\today}
\author{Michael Brodskiy\\ \small Professor: A. Martsinkovsky}

\begin{document}

\maketitle

\newpage

\tableofcontents

\newpage

\begin{itemize}

    \section{Vector Fields}

  \item A vector field in $\mathbb{R}^n$ is an assignment that goes from $\mathbb{R}^n\to\mathbb{R}^n$, where the former is viewed as a bunch of points, and the latter is a bunch of vectors

    \begin{itemize}

      \item Two examples: a force field (\textit{e.g.} gravity, electrostatic, magnetic) or a velocity field (\textit{e.g.} fluid mechanics)

    \end{itemize}

  \item \textit{Ex.} A gravitational field, with two masses, one fixed ($M$), and one floating ($m$)

    \begin{itemize}

      \item The pull $=$ magnitude $\cdot$ direction $\rightarrow G\dfrac{Mm}{|\overline{d}|^2}\cdot\dfrac{\overline{d}}{|\overline{d}|} = G\dfrac{Mm}{|\overline{d}|^3}\overline{d}$

    \end{itemize}

  \item \textit{Ex.} Gradient vector fields

    \begin{itemize}

      \item $f(\overline{x}) \rightarrow \nabla f = \Big\langle\dfrac{\delta f}{\delta x_1}, \dfrac{\delta f}{\delta x_2}, \cdots, \dfrac{\delta f}{\delta x_n}\Big\rangle$

      \item Not every vector field can be realized as a gradient vector field of some function

      \item The `del' operator is as follows: $\nabla :=\Big\langle \dfrac{\delta}{\delta x_1}, \dfrac{\delta}{\delta x_2}, \cdots, \dfrac{\delta}{\delta x_n} \Big\rangle$

      \item Apply $\nabla$ to a function $f$ to obtain a gradient vector field (use dot product)

      \item $\nabla \cdot F = \dfrac{\delta F_1}{\delta x_1} + \dfrac{\delta F_2}{\delta x_2} + \cdots + \dfrac{\delta F_n}{\delta x_n} = \text{div}(F)$

        \begin{itemize}

          \item This is the divergence of $F$

        \end{itemize}

      \item $\nabla \times F$ describes the curl of $F$ (how the vector field curls in three dimensions)

    \end{itemize}

    \begin{center}
      Using the $\nabla$ operator:\\
      \begin{tabular}{| l | l | l |}
        \hline
        input & output & significance\\
        \hline
        function $f$ & $\nabla f$ & gradient of $f$ (a vector field)\\
        \hline
        vector field of $f$ & $\nabla\cdot f$ & divergence of $f$ (a function)\\
        \hline
        $n=3$ vector field of $f$ & $\nabla\times f$ & curl of $f$ (a vector field)\\
        \hline
      \end{tabular}
    \end{center}

  \item To view 2D vector fields as 3D vector fields, convert $F=\langle P(x,y), Q(x,y)\rangle\to$\\$\langle P(x,y), Q(x,y), 0 \rangle$

  \item The curl of a two dimensional vector field, converted to three dimensions, is curl$(F) = Q_x-P_y$

    \section{Line Integrals}

  \item Can find Work, $W$, done by a force (a.k.a.\ a vector field) on an object along an oriented curve, $C$

  \item $W=\displaystyle \int_C \overline{F}\cdot dr$ — This is the line integral of $\overline{F}$ along the oriented curve $C$

  \item To compute the line integral, parametrize $C$, and reduce it to a Calculus II integral

    \begin{itemize}

      \item $d\overline{r} = \overline{r}'(t)\,dt$

      \item $\displaystyle \int_a^b \overline{F}(\overline{r}(t))\cdot \overline{r}'(t)\,dt$

    \end{itemize}

  \item Conservative vector fields are independent of path (taking any path from one point to another will always yield the same value)

\end{itemize}

\end{document}

