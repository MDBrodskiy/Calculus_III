%%%%%%%%%%%%%%%%%%%%%%%%%%%%%%%%%%%%%%%%%%%%%%%%%%%%%%%%%%%%%%%%%%%%%%%%%%%%%%%%%%%%%%%%%%%%%%%%%%%%%%%%%%%%%%%%%%%%%%%%%%%%%%%%%%%%%%%%%%%%%%%%%%%%%%%%%%%%%%%%%%%
% Written By Michael Brodskiy
% Class: Calculus III (MATH2321)
% Professor: A. Martsinkovsky
%%%%%%%%%%%%%%%%%%%%%%%%%%%%%%%%%%%%%%%%%%%%%%%%%%%%%%%%%%%%%%%%%%%%%%%%%%%%%%%%%%%%%%%%%%%%%%%%%%%%%%%%%%%%%%%%%%%%%%%%%%%%%%%%%%%%%%%%%%%%%%%%%%%%%%%%%%%%%%%%%%%

%%%%%%%%%%%%%%%%%%%%%%%%%%%%%%%%%%%%%%%%%%%%%%%%%%%%%%%%%%%%%%%%%%%%%%%%%%%%%%%%%%%%%%%%%%%%%%%%%%%%%%%%%%%%%%%%%%%%%%%%%%%%%%%%%%%%%%%%%%%%%%%%%%%%%%%%%%%%%%%%%%%
% Written By Michael Brodskiy
% Class: Calculus III (MATH2321)
% Professor: A. Martsinkovsky
%%%%%%%%%%%%%%%%%%%%%%%%%%%%%%%%%%%%%%%%%%%%%%%%%%%%%%%%%%%%%%%%%%%%%%%%%%%%%%%%%%%%%%%%%%%%%%%%%%%%%%%%%%%%%%%%%%%%%%%%%%%%%%%%%%%%%%%%%%%%%%%%%%%%%%%%%%%%%%%%%%%

\documentclass[12pt]{article} 
\usepackage{alphalph}
\usepackage[utf8]{inputenc}
\usepackage[russian,english]{babel}
\usepackage{titling}
\usepackage{amsmath}
\usepackage{graphicx}
\usepackage{enumitem}
\usepackage{amssymb}
\usepackage[super]{nth}
\usepackage{everysel}
\usepackage{ragged2e}
\usepackage{geometry}
\usepackage{multicol}
\usepackage{fancyhdr}
\usepackage{cancel}
\usepackage{siunitx}
\geometry{top=1.0in,bottom=1.0in,left=1.0in,right=1.0in}
\newcommand{\subtitle}[1]{%
  \posttitle{%
    \par\end{center}
    \begin{center}\large#1\end{center}
    \vskip0.5em}%

}
\usepackage{hyperref}
\hypersetup{
colorlinks=true,
linkcolor=blue,
filecolor=magenta,      
urlcolor=blue,
citecolor=blue,
}


\title{Section 1}
\date{\today}
\author{Michael Brodskiy\\ \small Professor: A. Martsinkovsky}

\begin{document}

\maketitle

\begin{itemize}

  \item What is a vector?

    \begin{itemize}

      \item A magnitude and a direction? (not all vectors in the real world can be added, so not entirely true)

      \item For our course, vectors exist in vector spaces ($\mathbb{R}^2,\,\mathbb{R}^3,\,\dots,\,\mathbb{R}^n$)

      \item $\overline{v}=\langle v_1,\,v_2,\,\dots,\,v_n \rangle$

      \item $\mathbb{R}^1$ represents scalars, while $\mathbb{R}^2,\,\mathbb{R}^3,\,\dots,\,\mathbb{R}^n$ are vectors

    \end{itemize}

  \item Properties of Vectors

    \begin{itemize}

      \item Can be added

        \begin{itemize}

          \item $\overline{v}=\langle v_1,\,v_2,\,\dots,\,v_n \rangle + \overline{w}=\langle w_1,\,w_2,\,\dots,\,w_n \rangle = \langle v_1+w_1,\,v_2+w_2,\,\dots,\,v_n+w_n \rangle$

          \item If forming a parallelogram from the vectors, the diagonal is the sum, $\overline{v}+\overline{w}$, of two vectors

        \end{itemize}

      \item Can be scaled (scalar multiplication)

        \begin{itemize}

          \item $2\overline{v}=\langle 2v_1,\,2v_2,\,\dots,\,2v_n \rangle$

          \item Magnitude is multiplied by the factor

        \end{itemize}

      \item Can find magnitude (length)

        \begin{itemize}

          \item $|\overline{v}|=\sqrt{v_1^2+v_2^2+\dots+v_n^2}$

          \item \textit{Ex.} $\overline{v}=\langle 2, -3 \rangle \Rightarrow |\overline{v}| = \sqrt{(2)^2 + (-3)^2)} = \sqrt{13}$

        \end{itemize}

      \item A vector divided by its own magnitude becomes a vector of magnitude 1 (unit vector)

        \begin{itemize}

          \item $|\frac{\overline{v}}{|\overline{v}|}|=1$

          \item Unit vectors are dimensionless (no units)

          \item A vector that is by itself of length 1 is not a unit vector

          \item A unit vector is simply a direction (all unit vectors from a given point form a circle)

        \end{itemize}

      \item Any non-zero vector is the product of its magnitude and its direction

        \begin{itemize}

          \item $\overline{v} = |\overline{v}| \cdot \frac{\overline{v}}{|\overline{v}|}$

        \end{itemize}

    \end{itemize}

  \item Linear Combinations

    \begin{itemize}

      \item $\overline{v}_1,\,\overline{v}_2,\,\dots,\,\overline{v}_s$

      \item A linear combination of $\overline{v}_i$ is any sum of the form $r_1\overline{v}_1+r_2\overline{v}_2+\dots+r_n\overline{v}_n$, where $r_i$ are scalars

    \end{itemize}

  \item Basis Vectors

    \begin{itemize}

      \item $\mathbb{R}^n$ standard basis vectors: $\overline{e}_1,\,\overline{e}_2,\,\dots,\,\overline{e}_n\Rightarrow\left\{\begin{array}{c} \overline{e}_1=\langle 1, 0, \dots, 0\rangle\\\overline{e}_2=\langle 0, 1, \dots, 0\rangle\\ \vdots\\ \overline{e}_n = \langle 0, 0, \dots, 1\rangle\end{array}$

      \item Any vector is a linear combination of the standard basis vectors

      \item $\overline{w}=\langle w_1,\,w_2,\,\dots,\,w_n\rangle= w_1\overline{e}_1 + w_2\overline{e}_2 + \dots + w_n\overline{e}_n$

      \item \textit{Ex.} $\overline{v}=\langle 2, -3 \rangle = 2\langle 1, 0 \rangle + -3 \langle 0, 1 \rangle$

    \end{itemize}

  \item Dot Product

    \begin{itemize}

      \item The dot product of two vectors is always a scalar

      \item Geometric Definition: $\overline{v} \cdot \overline{w} = |\overline{v}||\overline{w}|\cos(\theta)$, where $\theta$ is the angle between $\overline{v}$ and $\overline{w}$

        \begin{itemize}

          \item $\overline{v}\cdot\overline{w} = 0$ when $\theta = \frac{\pi}{2}$

          \item $\overline{v}\cdot\overline{w} > 0$ when $\theta$ is acute

          \item $\overline{v}\cdot\overline{w} < 0$ when $\theta$ is obtuse

        \end{itemize}

      \item Algebraic Definition: $\left\{\begin{array}{c} \overline{v} = \langle v_1,\,v_2,\,\dots,\,v_n \rangle \\ \overline{w} = \langle w_1,\,w_2,\,\dots,\,w_n \rangle \end{array} \Rightarrow \overline{v}\cdot\overline{w} = v_1w_1 + v_2w_2 + \dots + v_nw_n$

          \begin{itemize}

            \item \textit{Ex.} $\left\{\begin{array}{c} \overline{v} = \langle 4, 9, 5 \rangle \\ \overline{w} = \langle 4, 10, 3 \rangle \end{array} \Rightarrow \overline{v}\cdot\overline{w} = 4(4) + 9(10) + 5(3) = 121$

          \end{itemize}

        \item Together, the two definitions yield $\theta = \cos^{-1}\left(\frac{\overline{v}\cdot\overline{w}}{|\overline{v}||\overline{w}|}\right)$

          \begin{itemize}

            \item \textit{Ex.} Given $\overline{v}$ and $\overline{w}$ above, find the angle: $\cos^{-1}\left( \frac{121}{\sqrt{122}\sqrt{125}} \right)\approx .2$ rad

          \end{itemize}

      \item Vector Projection

        \begin{itemize}

          \item Assuming $\overline{u}$ is a unit vector, the projection of $\overline{F}$ onto $\overline{u}$ can be found using: $\text{proj}_{\overline{u}}\overline{F}=\left( \overline{F}\cdot\overline{u} \right)\overline{u}$

          \item In general, because $\overline{u}=\frac{\overline{v}}{|\overline{v}|}$, the formula becomes: $\text{proj}_{\overline{v}}\overline{F}=\left( \overline{F}\cdot\frac{\overline{v}}{|\overline{v}|} \right)\frac{\overline{v}}{|\overline{v}|}=\left( \frac{\overline{F}\cdot\overline{v}}{|\overline{v}|^2} \right)\overline{v}$

        \end{itemize}

    \end{itemize}

    \newpage

    \item Work

      \begin{itemize}

        \item $\overline{F}$ is a constant vector, $\overline{d}$ represents the displacement — work is defined as $\overline{F}\cdot\overline{d}$

        \item $W=\overline{F}\cdot\overline{d}=|\overline{F}||\overline{d}|\cos(\theta)$\\

        \begin{figure}[h!]
          \centering \tikzset{every picture/.style={line width=0.75pt}} %set default line width to 0.75pt        

\begin{tikzpicture}[x=0.75pt,y=0.75pt,yscale=-1,xscale=1]
%uncomment if require: \path (0,300); %set diagram left start at 0, and has height of 300

%Straight Lines [id:da3935293311213366] 
\draw    (238,106) -- (346.4,24.2) ;
\draw [shift={(348,23)}, rotate = 142.96] [color={rgb, 255:red, 0; green, 0; blue, 0 }  ][line width=0.75]    (10.93,-3.29) .. controls (6.95,-1.4) and (3.31,-0.3) .. (0,0) .. controls (3.31,0.3) and (6.95,1.4) .. (10.93,3.29)   ;
%Straight Lines [id:da4139500414390731] 
\draw    (238,106) -- (411,106) ;
\draw [shift={(413,106)}, rotate = 180] [color={rgb, 255:red, 0; green, 0; blue, 0 }  ][line width=0.75]    (10.93,-3.29) .. controls (6.95,-1.4) and (3.31,-0.3) .. (0,0) .. controls (3.31,0.3) and (6.95,1.4) .. (10.93,3.29)   ;
%Shape: Brace [id:dp7665000243293123] 
\draw   (239,108) .. controls (238.97,112.67) and (241.29,115.01) .. (245.96,115.04) -- (312.96,115.44) .. controls (319.63,115.48) and (322.95,117.83) .. (322.92,122.5) .. controls (322.95,117.83) and (326.29,115.52) .. (332.96,115.56)(329.96,115.54) -- (399.96,115.96) .. controls (404.63,115.99) and (406.97,113.67) .. (407,109) ;
%Straight Lines [id:da5936849259351216] 
\draw  [dash pattern={on 4.5pt off 4.5pt}]  (348,23) -- (348,106) ;
%Straight Lines [id:da5595911093922712] 
\draw  [dash pattern={on 4.5pt off 4.5pt}]  (238,25) -- (346,26) ;
%Straight Lines [id:da7278258054036395] 
\draw    (238,106) -- (238,27) ;
\draw [shift={(238,25)}, rotate = 90] [color={rgb, 255:red, 0; green, 0; blue, 0 }  ][line width=0.75]    (10.93,-3.29) .. controls (6.95,-1.4) and (3.31,-0.3) .. (0,0) .. controls (3.31,0.3) and (6.95,1.4) .. (10.93,3.29)   ;
%Shape: Arc [id:dp3843396654817173] 
\draw  [draw opacity=0] (271.07,81.06) .. controls (271.71,81.02) and (272.35,81) .. (273,81) .. controls (287.55,81) and (299.69,91.36) .. (302.42,105.12) -- (273,111) -- cycle ; \draw   (271.07,81.06) .. controls (271.71,81.02) and (272.35,81) .. (273,81) .. controls (287.55,81) and (299.69,91.36) .. (302.42,105.12) ;  

% Text Node
\draw (323.22,125.4) node [anchor=north] [inner sep=0.75pt]    {$\overline{d}$};
% Text Node
\draw (350,19.6) node [anchor=south west] [inner sep=0.75pt]    {$\overline{F}$};
% Text Node
\draw (236,34.4) node [anchor=north east] [inner sep=0.75pt]    {$\overline{F_{n}}$};
% Text Node
\draw (295,67.9) node [anchor=north west][inner sep=0.75pt]    {$\theta $};
% Text Node
\draw (415,109.4) node [anchor=north west][inner sep=0.75pt]    {$\overline{F_{d}}$};


\end{tikzpicture}

          \caption{Diagram of Work}
        \end{figure}

      \end{itemize}

    \item Lines and Planes

      \begin{itemize}

        \item \textit{Ex.} Given a point in $\mathbb{R}^2$, $( x_o, y_o ) = p$ and a vector $\overline{n}=\langle a, b \rangle$, find an equation of a line passing through $p$ and $\perp$ to $\overline{n}$

          \begin{figure}[h!]
            \centering \tikzset{every picture/.style={line width=0.75pt}} %set default line width to 0.75pt        

\begin{tikzpicture}[x=0.75pt,y=0.75pt,yscale=-1,xscale=1]
%uncomment if require: \path (0,300); %set diagram left start at 0, and has height of 300

%Shape: Boxed Line [id:dp04508569136084528] 
\draw    (231.29,156) -- (370.71,156) ;
\draw [shift={(372.71,156)}, rotate = 180] [color={rgb, 255:red, 0; green, 0; blue, 0 }  ][line width=0.75]    (10.93,-3.29) .. controls (6.95,-1.4) and (3.31,-0.3) .. (0,0) .. controls (3.31,0.3) and (6.95,1.4) .. (10.93,3.29)   ;
%Shape: Boxed Line [id:dp30544229715967797] 
\draw    (231.29,156) -- (231.29,16.58) ;
\draw [shift={(231.29,14.58)}, rotate = 90] [color={rgb, 255:red, 0; green, 0; blue, 0 }  ][line width=0.75]    (10.93,-3.29) .. controls (6.95,-1.4) and (3.31,-0.3) .. (0,0) .. controls (3.31,0.3) and (6.95,1.4) .. (10.93,3.29)   ;
%Shape: Boxed Line [id:dp9559818134426479] 
\draw    (89.87,156) -- (231.29,156) ;
%Shape: Circle [id:dp27693188797254353] 
\draw  [fill={rgb, 255:red, 0; green, 0; blue, 0 }  ,fill opacity=1 ] (226.29,156) .. controls (226.29,153.24) and (228.53,151) .. (231.29,151) .. controls (234.05,151) and (236.29,153.24) .. (236.29,156) .. controls (236.29,158.76) and (234.05,161) .. (231.29,161) .. controls (228.53,161) and (226.29,158.76) .. (226.29,156) -- cycle ;

% Text Node
\draw (229.29,17.98) node [anchor=north east] [inner sep=0.75pt]    {$\overline{n} =\langle a,b\rangle $};
% Text Node
\draw (231.29,159.4) node [anchor=north] [inner sep=0.75pt]    {$( x_{o} ,y_{o})$};
% Text Node
\draw (374.71,159.4) node [anchor=north west][inner sep=0.75pt]    {$( x,y)$};


\end{tikzpicture}

            \caption{Finding an Equation for a Line}
          \end{figure}

          \begin{itemize}

            \item Create a vector: $\langle x - x_o, y - y_o \rangle$, then, by definition, dot product becomes: $\langle a, b \rangle \cdot \langle x - x_o, y - y_o \rangle = 0$, which yields $ax + by - ax_o - by_o = 0$, which can be simplified to $ax+by+c=0$

          \end{itemize}

        \item In $\mathbb{R}^3$: $\langle a, b, c \rangle \cdot \langle x - x_o, y - y_o, z - z_o \rangle$ becomes $a(x-x_o) + b(y-y_o) + c(z-z_o) = 0$ and then $ax + by + cz + d = 0$, this forms a plane through point $p$ (in $\mathbb{R}^3$)

      \end{itemize}

    \item Parametric Description of a Line

      \begin{figure}[h!]
        \centering \tikzset{every picture/.style={line width=0.75pt}} %set default line width to 0.75pt        

\begin{tikzpicture}[x=0.75pt,y=0.75pt,yscale=-1,xscale=1]
%uncomment if require: \path (0,300); %set diagram left start at 0, and has height of 300

%Shape: Boxed Line [id:dp30544229715967797] 
\draw    (211.2,150.45) -- (389.13,83.7) ;
\draw [shift={(391,83)}, rotate = 159.44] [color={rgb, 255:red, 0; green, 0; blue, 0 }  ][line width=0.75]    (10.93,-3.29) .. controls (6.95,-1.4) and (3.31,-0.3) .. (0,0) .. controls (3.31,0.3) and (6.95,1.4) .. (10.93,3.29)   ;
%Shape: Boxed Line [id:dp030562383326971476] 
\draw    (235.29,207) -- (376.71,207) ;
%Shape: Boxed Line [id:dp8032447697238518] 
\draw    (235.29,207) -- (235.29,65.58) ;
%Curve Lines [id:da4708094973541954] 
\draw    (326,67) .. controls (312.28,91.5) and (336.02,77.58) .. (316.26,101.49) ;
\draw [shift={(315,103)}, rotate = 310.24] [color={rgb, 255:red, 0; green, 0; blue, 0 }  ][line width=0.75]    (10.93,-3.29) .. controls (6.95,-1.4) and (3.31,-0.3) .. (0,0) .. controls (3.31,0.3) and (6.95,1.4) .. (10.93,3.29)   ;
%Straight Lines [id:da8061489033654425] 
\draw    (235.29,207) -- (282.3,124.74) ;
\draw [shift={(283.29,123)}, rotate = 119.74] [color={rgb, 255:red, 0; green, 0; blue, 0 }  ][line width=0.75]    (10.93,-3.29) .. controls (6.95,-1.4) and (3.31,-0.3) .. (0,0) .. controls (3.31,0.3) and (6.95,1.4) .. (10.93,3.29)   ;
%Shape: Circle [id:dp5936029535008718] 
\draw  [fill={rgb, 255:red, 0; green, 0; blue, 0 }  ,fill opacity=1 ] (363,92) .. controls (363,89.79) and (364.79,88) .. (367,88) .. controls (369.21,88) and (371,89.79) .. (371,92) .. controls (371,94.21) and (369.21,96) .. (367,96) .. controls (364.79,96) and (363,94.21) .. (363,92) -- cycle ;
%Shape: Circle [id:dp8429171876576325] 
\draw  [fill={rgb, 255:red, 0; green, 0; blue, 0 }  ,fill opacity=1 ] (279.29,123) .. controls (279.29,120.79) and (281.08,119) .. (283.29,119) .. controls (285.5,119) and (287.29,120.79) .. (287.29,123) .. controls (287.29,125.21) and (285.5,127) .. (283.29,127) .. controls (281.08,127) and (279.29,125.21) .. (279.29,123) -- cycle ;

% Text Node
\draw (285.29,126.4) node [anchor=north west][inner sep=0.75pt]    {$( x_{o} ,y_{o})$};
% Text Node
\draw (328,63.6) node [anchor=south west] [inner sep=0.75pt]    {$\langle a,b\rangle $};
% Text Node
\draw (256,173.4) node [anchor=north west][inner sep=0.75pt]    {$\langle x_{o} ,y_{o} \rangle $};
% Text Node
\draw (369,95.4) node [anchor=north west][inner sep=0.75pt]    {$( x,y)$};


\end{tikzpicture}

        \caption{Parametrization}
      \end{figure}

      \begin{itemize}

        \item  $\left\{\begin{array}{c} x=x_o+ta\\y=y_o+tb\end{array}$

      \end{itemize}

\end{itemize}

\end{document}

