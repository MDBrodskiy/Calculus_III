%%%%%%%%%%%%%%%%%%%%%%%%%%%%%%%%%%%%%%%%%%%%%%%%%%%%%%%%%%%%%%%%%%%%%%%%%%%%%%%%%%%%%%%%%%%%%%%%%%%%%%%%%%%%%%%%%%%%%%%%%%%%%%%%%%%%%%%%%%%%%%%%%%%%%%%%%%%%%%%%%%%
% Written By Michael Brodskiy
% Class: Calculus III (MATH2321)
% Professor: A. Martsinkovsky
%%%%%%%%%%%%%%%%%%%%%%%%%%%%%%%%%%%%%%%%%%%%%%%%%%%%%%%%%%%%%%%%%%%%%%%%%%%%%%%%%%%%%%%%%%%%%%%%%%%%%%%%%%%%%%%%%%%%%%%%%%%%%%%%%%%%%%%%%%%%%%%%%%%%%%%%%%%%%%%%%%%

%%%%%%%%%%%%%%%%%%%%%%%%%%%%%%%%%%%%%%%%%%%%%%%%%%%%%%%%%%%%%%%%%%%%%%%%%%%%%%%%%%%%%%%%%%%%%%%%%%%%%%%%%%%%%%%%%%%%%%%%%%%%%%%%%%%%%%%%%%%%%%%%%%%%%%%%%%%%%%%%%%%
% Written By Michael Brodskiy
% Class: Calculus III (MATH2321)
% Professor: A. Martsinkovsky
%%%%%%%%%%%%%%%%%%%%%%%%%%%%%%%%%%%%%%%%%%%%%%%%%%%%%%%%%%%%%%%%%%%%%%%%%%%%%%%%%%%%%%%%%%%%%%%%%%%%%%%%%%%%%%%%%%%%%%%%%%%%%%%%%%%%%%%%%%%%%%%%%%%%%%%%%%%%%%%%%%%

\documentclass[12pt]{article} 
\usepackage{alphalph}
\usepackage[utf8]{inputenc}
\usepackage[russian,english]{babel}
\usepackage{titling}
\usepackage{amsmath}
\usepackage{graphicx}
\usepackage{enumitem}
\usepackage{amssymb}
\usepackage[super]{nth}
\usepackage{everysel}
\usepackage{ragged2e}
\usepackage{geometry}
\usepackage{multicol}
\usepackage{fancyhdr}
\usepackage{cancel}
\usepackage{siunitx}
\geometry{top=1.0in,bottom=1.0in,left=1.0in,right=1.0in}
\newcommand{\subtitle}[1]{%
  \posttitle{%
    \par\end{center}
    \begin{center}\large#1\end{center}
    \vskip0.5em}%

}
\usepackage{hyperref}
\hypersetup{
colorlinks=true,
linkcolor=blue,
filecolor=magenta,      
urlcolor=blue,
citecolor=blue,
}


\title{Section 3}
\date{\today}
\author{Michael Brodskiy\\ \small Professor: A. Martsinkovsky}

\begin{document}

\maketitle

\newpage

\tableofcontents

\newpage

\begin{itemize}

    \section{Partial Antiderivatives}

  \item \textit{Ex. } $\dfrac{\delta f}{\delta x} = 3x^2 - 5y^2\Rightarrow f = x^3 - 5y^2x + g(y)$

  $$\displaystyle \int_{y}^{y^2} 3x^2-5y^2\,dx\Rightarrow y^6 - 5y^4 - (y^3 - 5y^3)= y^6 - 5y^4 + 4y^3$$

  \vspace{5pt}

  $$\displaystyle \int_0^2\int_{y}^{y^2} 3x^2-5y^2\,dx\,dy\Rightarrow y^6 - 5y^4 - (y^3 - 5y^3) = \displaystyle \int_0^2 (y^6 - 5y^4 + 4y^3)\,dy =$$\vspace{-15pt}\\$$ \frac{1}{7}(2)^7-(2)^5+(2)^4 = \frac{128}{7} - 16 = \frac{16}{7}$$

  \item \textit{Ex. } $\displaystyle \int_1^3 \int_0^{\sin(x)} \frac{1+2y}{\sin(x)}\,dy\,dx=$

  $$\frac{1}{\sin(x)}\left( \sin(x)+(\sin(x))^2 \right) = \int_1^3 1 + \sin(x)\,dx = x - \cos(x) =$$\vspace{-15pt}\\$$3 - \cos(3) - 1 + \cos(1) = 2 + \cos(1) - \cos(3)$$

  \item \textit{Ex. } $\displaystyle \int_0^2\int_y^1\int_z^{yz} 8xyz\,dx\,dz\,dy=$

  $$yz((4(yz)^2 - 4z^2) = \int_0^2\int_y^1 4y^3z^3 - 4yz^3\,dz\,dy = (y^3 - y) - (y^7 - y^4) =$$\vspace{-15pt} \\$$\int_0^2 -y^7 + y^4 + y^3 - y\,dy = -\frac{1}{8}(2)^7 + \frac{1}{5}(2)^5 + \frac{1}{4}(2)^4 - \frac{1}{2}(2)^2 = -16 + \frac{32}{5} + 4 - 2 = -\frac{38}{5}$$

  \item \textit{Ex. } $\dfrac{\delta f}{\delta x} = 3x^2-5y^2,\,\,\dfrac{\delta f}{\delta y} = -10xy + 8y^3,\,\, f=$ ?

    $$\int \frac{\delta f}{\delta x}\,dx = x^3 - 5xy^2 + g(y) = \frac{\delta f}{\delta y} = -10xy + g'(y)\Rightarrow g'(y) = 8y^3 \Rightarrow f(x,y) = x^3-5xy^2+2y^4 + c$$

    \section{Integration in $\mathbb{R}^2$}

  \item Double Integral

    \begin{itemize}

      \item $\displaystyle \iint_R f(x,y)\,dA$

    \end{itemize}

  \item Fubini's Theorem: Utilize iterated integration to calculate multiple-integration

    \begin{itemize}

      \item $\displaystyle \iint_R f(x,y)\, dA \longrightarrow\int_a^b\int_c^d f(x,y)\,dy\,dx$

    \end{itemize}

  \item For Type I Regions:

    \begin{itemize}

      \item $\displaystyle \int_a^b\int_{p(x)}^{q(x)} f(x,y)\,dy\,dx$

      \item Occurs when $y$ is bounded by functions of $x$ and $x$ is bounded by vertical lines ($x=c$)

    \end{itemize}

  \item For Type II Regions:

    \begin{itemize}

      \item $\displaystyle \int_c^d\int_{r(x)}^{s(x)} f(x,y)\,dx\,dy$

      \item Occurs when $x$ is bounded by functions of $y$ and $y$ is bounded by horizontal lines ($y=c$)

    \end{itemize}

  \item A function can be Type I, Type II, both, or neither

  \item Regions can be broken down into parts to make calculations easier:

    \begin{itemize}

      \item Given $R$ and two subregions, $R'$ and $R''$, the integral becomes:

        \begin{itemize}

          \item $\displaystyle \iint_R f\,dA = \iint_{R'} f\,dA + \iint_{R''} f\,dA$

        \end{itemize}

    \end{itemize}

  \item Remark: If $f(x,y)=1$, then:

    \begin{itemize}

      \item $\displaystyle \iint_R\,dA =$ Area of $R$

    \end{itemize}

    \section{Integration with Polar Coordinates}

  \item Basic Properties $\left\{\begin{array}{c} x=r\cos(\theta)\\y=r\sin(\theta)\\x^2+y^2=r^2  \end{array}$

  \item $dA = r\,dr\,d\theta$

  \item $\displaystyle \iint_R f(x,y)\,dA=\displaystyle \iint_R f(r\cos(\theta), r\sin(\theta))r\,dr\,d\theta$

    \section{Integration in $\mathbb{R}^3$}

  \item $\displaystyle \iiint_S f(x,y,z)\,\,dV$, where $S\rightarrow\left\{\begin{array}{l} p(x,y) \leq z \leq q(x,y)\\ u(x) \leq y \leq v(x)\\ a \leq x \leq b  \end{array}$

    \item This can be expressed as: $\displaystyle \int_a^b\int_{u(x)}^{v(x)}\int_{p(x,y)}^{q(x,y)} f(x,y,z)\,\,dz\,dy\,dx$

      \section{Volume}

    \item $\displaystyle \iiint_S f(x,y,z)\, dV\Rightarrow f(x,y,z) = 1 \Rightarrow \iiint_S\,dV$ gives the volumes of solid $S $

      \section{Cylindrical and Spherical Coordinates}

    \item For cylindrical coordinates, refer to polar:

      \begin{itemize}

        \item $\left\{\begin{array}{l} x=r\cos(\theta)\\ y=r\sin(\theta)\\ z=z \end{array}$, and $dV=r\,dr\,d\theta\,dz$

          \item $0\leq r < \infty$, $0\leq\theta\leq2\pi$, and $-\infty< z < \infty$

      \end{itemize}

    \item Spherical coordinates use variables $\rho$, $\phi$, and $\theta$, where $\rho$ is the distance to the origin, $\phi$ is the vertical deviation (with respect to the positive $z$-axis), and $\theta$ is the horizontal deviation (or polar angle)

      \begin{itemize}

        \item $\left\{\begin{array}{l} x = \rho\sin(\phi)\cos(\theta)\\ y = \rho\sin(\phi)\sin(\theta)\\ z = \rho\cos(\phi)  \end{array}$, and $dV=\rho^2\sin(\phi)\,d\rho\,d\phi\,d\theta$

        \item It is important to remember the property $x^2 + y^2 + z^2 = \rho^2$

          \item $0\leq \rho < \infty$, $0\leq\theta\leq2\pi$, and $0 \leq \phi < \pi$

      \end{itemize}

\end{itemize}

\end{document}

