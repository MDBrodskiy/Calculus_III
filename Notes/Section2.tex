%%%%%%%%%%%%%%%%%%%%%%%%%%%%%%%%%%%%%%%%%%%%%%%%%%%%%%%%%%%%%%%%%%%%%%%%%%%%%%%%%%%%%%%%%%%%%%%%%%%%%%%%%%%%%%%%%%%%%%%%%%%%%%%%%%%%%%%%%%%%%%%%%%%%%%%%%%%%%%%%%%%
% Written By Michael Brodskiy
% Class: Calculus III (MATH2321)
% Professor: A. Martsinkovsky
%%%%%%%%%%%%%%%%%%%%%%%%%%%%%%%%%%%%%%%%%%%%%%%%%%%%%%%%%%%%%%%%%%%%%%%%%%%%%%%%%%%%%%%%%%%%%%%%%%%%%%%%%%%%%%%%%%%%%%%%%%%%%%%%%%%%%%%%%%%%%%%%%%%%%%%%%%%%%%%%%%%

%%%%%%%%%%%%%%%%%%%%%%%%%%%%%%%%%%%%%%%%%%%%%%%%%%%%%%%%%%%%%%%%%%%%%%%%%%%%%%%%%%%%%%%%%%%%%%%%%%%%%%%%%%%%%%%%%%%%%%%%%%%%%%%%%%%%%%%%%%%%%%%%%%%%%%%%%%%%%%%%%%%
% Written By Michael Brodskiy
% Class: Calculus III (MATH2321)
% Professor: A. Martsinkovsky
%%%%%%%%%%%%%%%%%%%%%%%%%%%%%%%%%%%%%%%%%%%%%%%%%%%%%%%%%%%%%%%%%%%%%%%%%%%%%%%%%%%%%%%%%%%%%%%%%%%%%%%%%%%%%%%%%%%%%%%%%%%%%%%%%%%%%%%%%%%%%%%%%%%%%%%%%%%%%%%%%%%

\documentclass[12pt]{article} 
\usepackage{alphalph}
\usepackage[utf8]{inputenc}
\usepackage[russian,english]{babel}
\usepackage{titling}
\usepackage{amsmath}
\usepackage{graphicx}
\usepackage{enumitem}
\usepackage{amssymb}
\usepackage[super]{nth}
\usepackage{everysel}
\usepackage{ragged2e}
\usepackage{geometry}
\usepackage{multicol}
\usepackage{fancyhdr}
\usepackage{cancel}
\usepackage{siunitx}
\geometry{top=1.0in,bottom=1.0in,left=1.0in,right=1.0in}
\newcommand{\subtitle}[1]{%
  \posttitle{%
    \par\end{center}
    \begin{center}\large#1\end{center}
    \vskip0.5em}%

}
\usepackage{hyperref}
\hypersetup{
colorlinks=true,
linkcolor=blue,
filecolor=magenta,      
urlcolor=blue,
citecolor=blue,
}


\title{Section 2}
\date{\today}
\author{Michael Brodskiy\\ \small Professor: A. Martsinkovsky}

\begin{document}

\maketitle

\newpage

\tableofcontents

\listoffigures

\newpage

\begin{itemize}

    \section{Partial Derivatives}

  \item The slope of $f(x,y)$ depends on the direction in the $xy$-plane

    \begin{itemize}

      \item The slope in the $x$-direction is called the partial derivative of $f$ with respect to $x$

      \item The slope in the $y$-direction is called the partial derivative of $f$ with respect to $y$

      \item Notation: $\dfrac{\partial f}{\partial x}$, $\dfrac{\partial f}{\partial y}$ or $f_x$, $f_y$

      \item For Second Derivatives: $\dfrac{\partial^2 f}{\partial x^2}$, $\dfrac{\partial^2 f}{\partial y^2}$, $\dfrac{\partial^2 f}{\partial y \partial x}$, $\dfrac{\partial^2 f}{\partial x \partial y}$ or $f_{xx}$, $f_{xy}$, $f_{yx}$, $f_{yy}$

        \vspace{10pt}

        \begin{center}
        If $f,f_x,f_y,$ and $f_{xy}$ are defined in a small disc around $(x_o,y_o)$ and $f_{yx}$ is continuous, then:\\ $f_{xy}=f_{yx}$\\in that disc
        \end{center}

    \end{itemize}

  \item The gradient of $f$

    \begin{itemize}

      \item Given $f(x_1,x_2,\dots,x_n)$, the gradient of $f$, $\overline{\nabla} f=\left\langle \dfrac{\partial f}{\partial x_1}, \dfrac{\partial f}{\partial x_2}, \dots, \dfrac{\partial f}{\partial x_n} \right\rangle$

      \item It can be computed at a point: $\overline{\nabla} f(p)=\left\langle \dfrac{\partial f}{\partial x_1}(p), \dfrac{\partial f}{\partial x_2}(p), \dots, \dfrac{\partial f}{\partial x_n}(p) \right\rangle$

      \item $\overline{\nabla} f\approx f'(x_o)\Delta x$

    \end{itemize}

    \section{Linear Approximation, Tangent Planes, and the Differential}

  \item In calculus I, the linear approximation is given by: $f(x) \approx f(a) + f'(a)(x-a)$

  \item  In calculus III, the approximation uses the gradient: $\Delta f\approx \overline{\nabla} f(p) \cdot \Delta \overline{x}$

  \item \textit{Ex.\ in $\mathbb{R}^2$} $z=f(x,y)$, $p=(a,b)$:

    \vspace{-20pt}

    $$\Delta f \approx \overline{\nabla}f(a,b) \cdot \langle x-a, y-b \rangle \Rightarrow f(x,y) - f(a,b) \Rightarrow f(a,b) + f_x(a,b)(x-a) + f_y(a,b)(y-b)$$

    \vspace{-20pt}

    \begin{center}
      Thus:
    \end{center}

    \vspace{-25pt}

    $$f(x,y) \approx f(a,b) + f_x(a,b)(x-a) + f_y(a,b)(y-b)$$

    \vspace{-15pt}

    \begin{center}
      We find the linearization of $f(x,y)$ near $(a,b)$ (in $\mathbb{R}^2$)
    \end{center}

  \item Linearization of $f(x,y)$ near $(a,b)$ is denoted by $L_f(\overline{x},\overline{p})$, where $p$ is the vector $\langle a, b \rangle$

  \item \textit{Ex. Given a cylinder of radius $r=.5$ and a height of $h=1$, estimate the change in volume when the radius is increased by $.1$ and the height is decreased by $.1$}

    $$V=\pi r^2h\Rightarrow \Delta V \approx 2\pi(.5)(1)(r - .5) + \pi(.5)^2(h - 1) = \pi(r - .5) + .25\pi(h - 1) \approx$$
    $$-.75\pi + \pi r + .25\pi h\Rightarrow -.75\pi + \pi(.6) + .25\pi(.9) = .075\pi$$

    \vspace{5pt}

  \item The graph of $z=L_f(\overline{x},\overline{p})$ is called the tangent set to $f$ at $p$

  \item Differentials

    \begin{itemize}

      \item $df = \overline{\nabla}f\cdot d\overline{x}$

      \item $df = f_xdx + f_ydy + f_zdz$

      \item $df = \dfrac{\partial f}{\partial x_1}dx_1 + \dfrac{\partial f}{\partial x_2}dx_2 + \dots + \dfrac{\partial f}{\partial x_n}dx_n$

    \end{itemize}

  \item Relative Differentials

    \begin{itemize}

      \item $\dfrac{df}{f}$

      \item Think of this as a stencil for relative error $\left( \dfrac{\Delta f}{f} \right)$

    \end{itemize}

    \section{Differentiation Rules}

  \item Linearity of differentiation

    \begin{enumerate}

      \item $\nabla(af\pm bg) = a\nabla f \pm b\nabla g$

    \end{enumerate}

  \item Product rule

    \begin{enumerate}

        \setcounter{enumi}{1}

      \item $\nabla (fg)(p)= \nabla f(p)g(p) = \nabla g(p) f(p)$

    \end{enumerate}

  \item Quotient rule

    \begin{enumerate}

        \setcounter{enumi}{2}

      \item $\nabla\left( \dfrac{f}{g} \right)\Big |_p=\dfrac{g(p)\nabla f(p) - f(p) \nabla g(p)}{g^2(p)}$

    \end{enumerate}

  \item Power rule

    \begin{enumerate}

        \setcounter{enumi}{3}

      \item $\nabla f^{\alpha}(p) = \alpha f^{\alpha - 1}(p) \nabla f(p)$

    \end{enumerate}

    \newpage

  \item Chain rule

    \begin{enumerate}

        \setcounter{enumi}{4}

      \item $\dfrac{df}{dt}=\dfrac{\partial f}{\partial x}\dfrac{dx}{dt} + \dfrac{\partial f}{\partial y}\dfrac{dy}{dt} + \dots + \dfrac{\partial f}{\partial z}\dfrac{dz}{dt}\Rightarrow \nabla f \cdot \dfrac{d\overline{x}}{dt}$

    \end{enumerate}

  \item \textit{Ex.} A particle is moving through space. At $t=2$ seconds, the particle is at $(3,4,7)$, and is moving with velocity $\langle -2, 1, 5 \rangle$ meters per second. Suppose that there is also an electric potential in space, given by $\phi(x,y,z)=xy-z^2$ volts. Find the instantaneous rate of change, $w$, $r$, $t$, time $t$, of the electric potential at the particle's position at $t=2$ seconds.

    $$\dfrac{d\phi}{dt}\Big|_{t=2}=\nabla \phi(\langle 3, 4, 7 \rangle)\cdot \dfrac{d\overline{p}}{dt}\Big|_{t=2}= \nabla \phi(\langle 3, 4, 7 \rangle)\cdot v(2)\Rightarrow$$

    $$\nabla \phi = \langle y, x, -2z \rangle(\langle 3, 4, 7 \rangle) = \langle 4, 3, -14 \rangle \Rightarrow \langle 4, 3, -14 \rangle \cdot \langle -2, 1, 5 \rangle = -75 \text{ volts per second}$$

\end{itemize}

\end{document}

