%%%%%%%%%%%%%%%%%%%%%%%%%%%%%%%%%%%%%%%%%%%%%%%%%%%%%%%%%%%%%%%%%%%%%%%%%%%%%%%%%%%%%%%%%%%%%%%%%%%%%%%%%%%%%%%%%%%%%%%%%%%%%%%%%%%%%%%%%%%%%%%%%%%%%%%%%%%%%%%%%%%
% Written By Michael Brodskiy
% Class: Cornerstone Engineering 1 & 2 (GE1501 & GE1502)
% Professor: B. O'Connell
%%%%%%%%%%%%%%%%%%%%%%%%%%%%%%%%%%%%%%%%%%%%%%%%%%%%%%%%%%%%%%%%%%%%%%%%%%%%%%%%%%%%%%%%%%%%%%%%%%%%%%%%%%%%%%%%%%%%%%%%%%%%%%%%%%%%%%%%%%%%%%%%%%%%%%%%%%%%%%%%%%%

\documentclass[12pt]{article} 
\usepackage{alphalph}
\usepackage[utf8]{inputenc}
\usepackage[russian,english]{babel}
\usepackage{titling}
\usepackage{amsmath}
\usepackage{graphicx}
\usepackage{enumitem}
\usepackage{amssymb}
\usepackage[super]{nth}
\usepackage{everysel}
\usepackage{ragged2e}
\usepackage{geometry}
\usepackage{multicol}
\usepackage{fancyhdr}
\usepackage{cancel}
\usepackage{siunitx}
\geometry{top=1.0in,bottom=1.0in,left=1.0in,right=1.0in}
\newcommand{\subtitle}[1]{%
  \posttitle{%
    \par\end{center}
    \begin{center}\large#1\end{center}
    \vskip0.5em}%

}
\usepackage{hyperref}
\hypersetup{
colorlinks=true,
linkcolor=blue,
filecolor=magenta,      
urlcolor=blue,
citecolor=blue,
}


\title{Notes 1}
\date{\today}
\author{Michael Brodskiy\\ \small Professor: A. Martsinkovsky}

\begin{document}

\maketitle

\begin{itemize}

  \item What is a vector?

    \begin{itemize}

      \item A magnitude and a direction? (not all vectors in the real world can be added, so not entirely true)

      \item For our course, vectors exist in vector spaces ($\mathbb{R}^2,\,\mathbb{R}^3,\,\dots,\,\mathbb{R}^n$)

      \item $\overline{v}=\langle v_1,\,v_2,\,\dots,\,v_n \rangle$

      \item $\mathbb{R}^1$ represents scalars, while $\mathbb{R}^2,\,\mathbb{R}^3,\,\dots,\,\mathbb{R}^n$ are vectors

    \end{itemize}

  \item Properties of Vectors

    \begin{itemize}

      \item Can be added

        \begin{itemize}

          \item $\overline{v}=\langle v_1,\,v_2,\,\dots,\,v_n \rangle + \overline{w}=\langle w_1,\,w_2,\,\dots,\,w_n \rangle = \langle v_1+w_1,\,v_2+w_2,\,\dots,\,v_n+w_n \rangle$

          \item If forming a parallelogram from the vectors, the diagonal is the sum, $\overline{v}+\overline{w}$, of two vectors

        \end{itemize}

      \item Can be scaled (scalar multiplication)

        \begin{itemize}

          \item $2\overline{v}=\langle 2v_1,\,2v_2,\,\dots,\,2v_n \rangle$

          \item Magnitude is multiplied by the factor

        \end{itemize}

      \item Can find magnitude (length)

        \begin{itemize}

          \item $|\overline{v}|=\sqrt{v_1^2+v_2^2+\dots+v_n^2}$

          \item \textit{Ex.} $\overline{v}=\langle 2, -3 \rangle \Rightarrow |\overline{v}| = \sqrt{(2)^2 + (-3)^2)} = \sqrt{13}$

        \end{itemize}

      \item A vector divided by its own magnitude becomes a vector of magnitude 1 (unit vector)

        \begin{itemize}

          \item $|\frac{\overline{v}}{|\overline{v}|}|=1$

          \item Unit vectors are dimensionless (no units)

          \item A vector that is by itself of length 1 is not a unit vector

          \item A unit vector is simply a direction (all unit vectors from a given point form a circle)

        \end{itemize}

      \item Any non-zero vector is the product of its magnitude and its direction

        \begin{itemize}

          \item $\overline{v} = |\overline{v}| \cdot \frac{\overline{v}}{|\overline{v}|}$

        \end{itemize}

    \end{itemize}

  \item Linear Combinations

    \begin{itemize}

      \item $\overline{v}_1,\,\overline{v}_2,\,\dots,\,\overline{v}_s$

      \item A linear combination of $\overline{v}_i$ is any sum of the form $r_1\overline{v}_1+r_2\overline{v}_2+\dots+r_n\overline{v}_n$, where $r_i$ are scalars

    \end{itemize}

  \item Basis Vectors

    \begin{itemize}

      \item $\mathbb{R}^n$ standard basis vectors: $\overline{e}_1,\,\overline{e}_2,\,\dots,\,\overline{e}_n\Rightarrow\left\{\begin{array}{c} \overline{e}_1=\langle 1, 0, \dots, 0\rangle\\\overline{e}_2=\langle 0, 1, \dots, 0\rangle\\ \vdots\\ \overline{e}_n = \langle 0, 0, \dots, 1\rangle\end{array}$

      \item Any vector is a linear combination of the standard basis vectors

      \item $\overline{w}=\langle w_1,\,w_2,\,\dots,\,w_n\rangle= w_1\overline{e}_1 + w_2\overline{e}_2 + \dots + w_n\overline{e}_n$

      \item \textit{Ex.} $\overline{v}=\langle 2, -3 \rangle = 2\langle 1, 0 \rangle + -3 \langle 0, 1 \rangle$

    \end{itemize}

  \item Dot Product

    \begin{itemize}

      \item The dot product of two vectors is always a scalar

      \item Geometric Definition: $\overline{v} \cdot \overline{w} = |\overline{v}||\overline{w}|\cos(\theta)$, where $\theta$ is the angle between $\overline{v}$ and $\overline{w}$

        \begin{itemize}

          \item $\overline{v}\cdot\overline{w} = 0$ when $\theta = \frac{\pi}{2}$

          \item $\overline{v}\cdot\overline{w} > 0$ when $\theta$ is acute

          \item $\overline{v}\cdot\overline{w} < 0$ when $\theta$ is obtuse

        \end{itemize}

    \end{itemize}

\end{itemize}

\end{document}

