%%%%%%%%%%%%%%%%%%%%%%%%%%%%%%%%%%%%%%%%%%%%%%%%%%%%%%%%%%%%%%%%%%%%%%%%%%%%%%%%%%%%%%%%%%%%%%%%%%%%%%%%%%%%%%%%%%%%%%%%%%%%%%%%%%%%%%%%%%%%%%%%%%%%%%%%%%%%%%%%%%%
% Written By Michael Brodskiy
% Class: Calculus III (MATH2321)
% Professor: A. Martsinkovsky
%%%%%%%%%%%%%%%%%%%%%%%%%%%%%%%%%%%%%%%%%%%%%%%%%%%%%%%%%%%%%%%%%%%%%%%%%%%%%%%%%%%%%%%%%%%%%%%%%%%%%%%%%%%%%%%%%%%%%%%%%%%%%%%%%%%%%%%%%%%%%%%%%%%%%%%%%%%%%%%%%%%

%%%%%%%%%%%%%%%%%%%%%%%%%%%%%%%%%%%%%%%%%%%%%%%%%%%%%%%%%%%%%%%%%%%%%%%%%%%%%%%%%%%%%%%%%%%%%%%%%%%%%%%%%%%%%%%%%%%%%%%%%%%%%%%%%%%%%%%%%%%%%%%%%%%%%%%%%%%%%%%%%%%
% Written By Michael Brodskiy
% Class: Calculus III (MATH2321)
% Professor: A. Martsinkovsky
%%%%%%%%%%%%%%%%%%%%%%%%%%%%%%%%%%%%%%%%%%%%%%%%%%%%%%%%%%%%%%%%%%%%%%%%%%%%%%%%%%%%%%%%%%%%%%%%%%%%%%%%%%%%%%%%%%%%%%%%%%%%%%%%%%%%%%%%%%%%%%%%%%%%%%%%%%%%%%%%%%%

\documentclass[12pt]{article} 
\usepackage{alphalph}
\usepackage[utf8]{inputenc}
\usepackage[russian,english]{babel}
\usepackage{titling}
\usepackage{amsmath}
\usepackage{graphicx}
\usepackage{enumitem}
\usepackage{amssymb}
\usepackage[super]{nth}
\usepackage{everysel}
\usepackage{ragged2e}
\usepackage{geometry}
\usepackage{multicol}
\usepackage{fancyhdr}
\usepackage{cancel}
\usepackage{siunitx}
\geometry{top=1.0in,bottom=1.0in,left=1.0in,right=1.0in}
\newcommand{\subtitle}[1]{%
  \posttitle{%
    \par\end{center}
    \begin{center}\large#1\end{center}
    \vskip0.5em}%

}
\usepackage{hyperref}
\hypersetup{
colorlinks=true,
linkcolor=blue,
filecolor=magenta,      
urlcolor=blue,
citecolor=blue,
}


\title{Homework 1.4}
\date{\today}
\author{Michael Brodskiy\\ \small Professor: A. Martsinkovsky}

\begin{document}

\maketitle

\begin{enumerate}

  \item $\overline{v}_{\parallel} = ( -5, 3 ) - ( 1, 2 ) = \langle -6, 1 \rangle\Rightarrow \mathbb{L}(t) = (-5, 3) + t\langle -6, 1 \rangle \Rightarrow \frac{x + 5}{-6} = y - 3$

  \item $\overline{v}_{\parallel} = ( 0, 1 ) - ( 1, 0 ) = \langle -1, 1 \rangle \Rightarrow \mathbb{L}(t) = (0, 1) + t\langle -1, 1 \rangle\Rightarrow -x = y - 1$

  \item $\overline{v}_{\parallel} = (1, 2, 3) - (3, 2, 1) = \langle -2, 0, 2 \rangle \Rightarrow \mathbb{L}(t) = (1, 2, 3) + t\langle -2, 0, 2\rangle\Rightarrow -\frac{x - 1}{2} = \frac{z - 3}{2}$

  \item $\overline{v}_{\parallel} = (1, 0, 0) - (0, -1, -2) = \langle 1, 1, 2 \rangle \Rightarrow (1, 0, 0) + t\langle 1, 1, 2\rangle \Rightarrow$ Symmetric equations are not possible

    \setcounter{enumi}{8}

  \item

  \item

  \item

  \item

  \item

  \item

  \item

  \item

  \item

    \setcounter{enumi}{18}

  \item

    \setcounter{enumi}{20}

  \item

  \item

  \item

    \setcounter{enumi}{26}

  \item

  \item

  \item

  \item

\end{enumerate}

\end{document}

