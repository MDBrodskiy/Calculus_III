%%%%%%%%%%%%%%%%%%%%%%%%%%%%%%%%%%%%%%%%%%%%%%%%%%%%%%%%%%%%%%%%%%%%%%%%%%%%%%%%%%%%%%%%%%%%%%%%%%%%%%%%%%%%%%%%%%%%%%%%%%%%%%%%%%%%%%%%%%%%%%%%%%%%%%%%%%%%%%%%%%%
% Written By Michael Brodskiy
% Class: Calculus III (MATH2321)
% Professor: A. Martsinkovsky
%%%%%%%%%%%%%%%%%%%%%%%%%%%%%%%%%%%%%%%%%%%%%%%%%%%%%%%%%%%%%%%%%%%%%%%%%%%%%%%%%%%%%%%%%%%%%%%%%%%%%%%%%%%%%%%%%%%%%%%%%%%%%%%%%%%%%%%%%%%%%%%%%%%%%%%%%%%%%%%%%%%

%%%%%%%%%%%%%%%%%%%%%%%%%%%%%%%%%%%%%%%%%%%%%%%%%%%%%%%%%%%%%%%%%%%%%%%%%%%%%%%%%%%%%%%%%%%%%%%%%%%%%%%%%%%%%%%%%%%%%%%%%%%%%%%%%%%%%%%%%%%%%%%%%%%%%%%%%%%%%%%%%%%
% Written By Michael Brodskiy
% Class: Calculus III (MATH2321)
% Professor: A. Martsinkovsky
%%%%%%%%%%%%%%%%%%%%%%%%%%%%%%%%%%%%%%%%%%%%%%%%%%%%%%%%%%%%%%%%%%%%%%%%%%%%%%%%%%%%%%%%%%%%%%%%%%%%%%%%%%%%%%%%%%%%%%%%%%%%%%%%%%%%%%%%%%%%%%%%%%%%%%%%%%%%%%%%%%%

\documentclass[12pt]{article} 
\usepackage{alphalph}
\usepackage[utf8]{inputenc}
\usepackage[russian,english]{babel}
\usepackage{titling}
\usepackage{amsmath}
\usepackage{graphicx}
\usepackage{enumitem}
\usepackage{amssymb}
\usepackage[super]{nth}
\usepackage{everysel}
\usepackage{ragged2e}
\usepackage{geometry}
\usepackage{multicol}
\usepackage{fancyhdr}
\usepackage{cancel}
\usepackage{siunitx}
\geometry{top=1.0in,bottom=1.0in,left=1.0in,right=1.0in}
\newcommand{\subtitle}[1]{%
  \posttitle{%
    \par\end{center}
    \begin{center}\large#1\end{center}
    \vskip0.5em}%

}
\usepackage{hyperref}
\hypersetup{
colorlinks=true,
linkcolor=blue,
filecolor=magenta,      
urlcolor=blue,
citecolor=blue,
}


\title{Homework 1.2}
\date{\today}
\author{Michael Brodskiy\\ \small Professor: A. Martsinkovsky}

\begin{document}

\maketitle

\begin{enumerate}

  \item 

    \begin{itemize}

      \item Magnitude: $\sqrt{3^2 + 4^2}=5$ feet per second

      \item Direction: $\frac{1}{5}\langle 3, 4 \rangle = \langle \frac{3}{5}, \frac{4}{5} \rangle$

    \end{itemize}

    \setcounter{enumi}{2}

  \item

    \begin{itemize}

      \item Magnitude: $\sqrt{(-6)^2 + (1)^2 + (6)^2}=\sqrt{73}$ meters per second

      \item Direction: $\frac{1}{\sqrt{73}}\langle -6, 1, 6 \rangle=\langle \frac{-6}{\sqrt{73}}, \frac{1}{\sqrt{73}}, \frac{6}{\sqrt{73}}\rangle$

    \end{itemize}

    \setcounter{enumi}{4}

  \item

    \begin{itemize}

      \item Magnitude: $\sqrt{(1)^2 + (-1)^2 + (1)^2 + (-1)^2} = 2$

      \item Direction: $\frac{1}{2}\langle 1, -1, 1, -1 \rangle=\langle \frac{1}{2}, -\frac{1}{2}, \frac{1}{2}, -\frac{1}{2}\rangle$

    \end{itemize}

    \setcounter{enumi}{6}

  \item

    \begin{itemize}

      \item Magnitude: $\sqrt{(2)^2 + (-3)^2 + (1)^2}=\sqrt{14}$

      \item Direction: $\frac{1}{14}(2\bold{i}-3\bold{j}+\bold{k}) = \frac{2}{14}\bold{i} - \frac{3}{14}\bold{j} + \frac{1}{14}\bold{k}$

    \end{itemize}

    \setcounter{enumi}{8}

  \item See figure below:

    \begin{figure}[H]
      \centering \tikzset{every picture/.style={line width=0.75pt}} %set default line width to 0.75pt        

\begin{tikzpicture}[x=0.75pt,y=0.75pt,yscale=-1,xscale=1]
%uncomment if require: \path (0,423); %set diagram left start at 0, and has height of 423

%Straight Lines [id:da7905086094561125] 
\draw [color={rgb, 255:red, 208; green, 2; blue, 27 }  ,draw opacity=1 ]   (215.6,324.8) -- (374.49,87.66) ;
\draw [shift={(375.6,86)}, rotate = 123.82] [color={rgb, 255:red, 208; green, 2; blue, 27 }  ,draw opacity=1 ][line width=0.75]    (10.93,-3.29) .. controls (6.95,-1.4) and (3.31,-0.3) .. (0,0) .. controls (3.31,0.3) and (6.95,1.4) .. (10.93,3.29)   ;
%Shape: Axis 2D [id:dp7962121836036462] 
\draw  (186,324.8) -- (482,324.8)(215.6,53) -- (215.6,355) (475,319.8) -- (482,324.8) -- (475,329.8) (210.6,60) -- (215.6,53) -- (220.6,60)  ;
%Shape: Grid [id:dp7870996256892415] 
\draw  [draw opacity=0] (215.6,86) -- (455.6,86) -- (455.6,326) -- (215.6,326) -- cycle ; \draw  [color={rgb, 255:red, 155; green, 155; blue, 155 }  ,draw opacity=0.5 ] (255.6,86) -- (255.6,326)(295.6,86) -- (295.6,326)(335.6,86) -- (335.6,326)(375.6,86) -- (375.6,326)(415.6,86) -- (415.6,326) ; \draw  [color={rgb, 255:red, 155; green, 155; blue, 155 }  ,draw opacity=0.5 ] (215.6,126) -- (455.6,126)(215.6,166) -- (455.6,166)(215.6,206) -- (455.6,206)(215.6,246) -- (455.6,246)(215.6,286) -- (455.6,286) ; \draw  [color={rgb, 255:red, 155; green, 155; blue, 155 }  ,draw opacity=0.5 ] (215.6,86) -- (455.6,86) -- (455.6,326) -- (215.6,326) -- cycle ;
%Straight Lines [id:da6128088926831519] 
\draw [color={rgb, 255:red, 74; green, 144; blue, 226 }  ,draw opacity=1 ]   (255.6,246) -- (294.19,207.41) ;
\draw [shift={(295.6,206)}, rotate = 135] [color={rgb, 255:red, 74; green, 144; blue, 226 }  ,draw opacity=1 ][line width=0.75]    (10.93,-3.29) .. controls (6.95,-1.4) and (3.31,-0.3) .. (0,0) .. controls (3.31,0.3) and (6.95,1.4) .. (10.93,3.29)   ;
%Straight Lines [id:da9944692355606897] 
\draw [color={rgb, 255:red, 74; green, 144; blue, 226 }  ,draw opacity=1 ]   (215.6,324.8) -- (294.48,207.66) ;
\draw [shift={(295.6,206)}, rotate = 123.96] [color={rgb, 255:red, 74; green, 144; blue, 226 }  ,draw opacity=1 ][line width=0.75]    (10.93,-3.29) .. controls (6.95,-1.4) and (3.31,-0.3) .. (0,0) .. controls (3.31,0.3) and (6.95,1.4) .. (10.93,3.29)   ;
%Straight Lines [id:da7148545358160481] 
\draw [color={rgb, 255:red, 208; green, 2; blue, 27 }  ,draw opacity=1 ]   (215.6,326) -- (294.71,167.79) ;
\draw [shift={(295.6,166)}, rotate = 116.57] [color={rgb, 255:red, 208; green, 2; blue, 27 }  ,draw opacity=1 ][line width=0.75]    (10.93,-3.29) .. controls (6.95,-1.4) and (3.31,-0.3) .. (0,0) .. controls (3.31,0.3) and (6.95,1.4) .. (10.93,3.29)   ;
%Straight Lines [id:da27745396609468553] 
\draw [color={rgb, 255:red, 74; green, 144; blue, 226 }  ,draw opacity=1 ]   (215.6,326) -- (254.71,247.79) ;
\draw [shift={(255.6,246)}, rotate = 116.57] [color={rgb, 255:red, 74; green, 144; blue, 226 }  ,draw opacity=1 ][line width=0.75]    (10.93,-3.29) .. controls (6.95,-1.4) and (3.31,-0.3) .. (0,0) .. controls (3.31,0.3) and (6.95,1.4) .. (10.93,3.29)   ;
%Straight Lines [id:da07820003545479248] 
\draw [color={rgb, 255:red, 208; green, 2; blue, 27 }  ,draw opacity=1 ]   (295.6,166) -- (374.19,87.41) ;
\draw [shift={(375.6,86)}, rotate = 135] [color={rgb, 255:red, 208; green, 2; blue, 27 }  ,draw opacity=1 ][line width=0.75]    (10.93,-3.29) .. controls (6.95,-1.4) and (3.31,-0.3) .. (0,0) .. controls (3.31,0.3) and (6.95,1.4) .. (10.93,3.29)   ;

% Text Node
\draw (253.6,246.1) node [anchor=south east] [inner sep=0.75pt]  [color={rgb, 255:red, 74; green, 144; blue, 226 }  ,opacity=1 ]  {$\overline{v}$};
% Text Node
\draw (295.6,202.6) node [anchor=south] [inner sep=0.75pt]  [color={rgb, 255:red, 74; green, 144; blue, 226 }  ,opacity=1 ]  {$\overline{w}$};
% Text Node
\draw (257.6,268.8) node [anchor=north west][inner sep=0.75pt]  [color={rgb, 255:red, 74; green, 144; blue, 226 }  ,opacity=1 ]  {$\overline{v} +\overline{w}$};
% Text Node
\draw (295.6,162.6) node [anchor=south] [inner sep=0.75pt]  [color={rgb, 255:red, 208; green, 2; blue, 27 }  ,opacity=1 ]  {$a\overline{v}$};
% Text Node
\draw (377.6,89.4) node [anchor=north west][inner sep=0.75pt]  [color={rgb, 255:red, 208; green, 2; blue, 27 }  ,opacity=1 ]  {$a\overline{w}$};
% Text Node
\draw (325,168.73) node [anchor=north west][inner sep=0.75pt]  [color={rgb, 255:red, 208; green, 2; blue, 27 }  ,opacity=1 ]  {$a\overline{v} +a\overline{w}$};


\end{tikzpicture}

      \caption{$\overline{v} = \langle 1, 2\rangle,\, \overline{w} = \langle 1, 1 \rangle,\, a = 2$}
    \end{figure}

  \item See figure below:

    \begin{figure}[H]
      \centering \tikzset{every picture/.style={line width=0.75pt}} %set default line width to 0.75pt        

\begin{tikzpicture}[x=0.75pt,y=0.75pt,yscale=-1,xscale=1]
%uncomment if require: \path (0,682); %set diagram left start at 0, and has height of 682

%Shape: Boxed Line [id:dp772165313436463] 
\draw [color={rgb, 255:red, 208; green, 2; blue, 27 }  ,draw opacity=1 ]   (348.6,277.8) -- (189.71,514.94) ;
\draw [shift={(188.6,516.6)}, rotate = 303.82] [color={rgb, 255:red, 208; green, 2; blue, 27 }  ,draw opacity=1 ][line width=0.75]    (10.93,-3.29) .. controls (6.95,-1.4) and (3.31,-0.3) .. (0,0) .. controls (3.31,0.3) and (6.95,1.4) .. (10.93,3.29)   ;
%Shape: Axis 2D [id:dp7962121836036462] 
\draw  (319,277.8) -- (615,277.8)(348.6,6) -- (348.6,308) (608,272.8) -- (615,277.8) -- (608,282.8) (343.6,13) -- (348.6,6) -- (353.6,13)  ;
%Shape: Grid [id:dp7870996256892415] 
\draw  [draw opacity=0] (348.6,37.8) -- (588.6,37.8) -- (588.6,277.8) -- (348.6,277.8) -- cycle ; \draw  [color={rgb, 255:red, 155; green, 155; blue, 155 }  ,draw opacity=0.5 ] (388.6,37.8) -- (388.6,277.8)(428.6,37.8) -- (428.6,277.8)(468.6,37.8) -- (468.6,277.8)(508.6,37.8) -- (508.6,277.8)(548.6,37.8) -- (548.6,277.8) ; \draw  [color={rgb, 255:red, 155; green, 155; blue, 155 }  ,draw opacity=0.5 ] (348.6,77.8) -- (588.6,77.8)(348.6,117.8) -- (588.6,117.8)(348.6,157.8) -- (588.6,157.8)(348.6,197.8) -- (588.6,197.8)(348.6,237.8) -- (588.6,237.8) ; \draw  [color={rgb, 255:red, 155; green, 155; blue, 155 }  ,draw opacity=0.5 ] (348.6,37.8) -- (588.6,37.8) -- (588.6,277.8) -- (348.6,277.8) -- cycle ;
%Straight Lines [id:da6128088926831519] 
\draw [color={rgb, 255:red, 74; green, 144; blue, 226 }  ,draw opacity=1 ]   (388.6,199) -- (427.19,160.41) ;
\draw [shift={(428.6,159)}, rotate = 135] [color={rgb, 255:red, 74; green, 144; blue, 226 }  ,draw opacity=1 ][line width=0.75]    (10.93,-3.29) .. controls (6.95,-1.4) and (3.31,-0.3) .. (0,0) .. controls (3.31,0.3) and (6.95,1.4) .. (10.93,3.29)   ;
%Straight Lines [id:da9944692355606897] 
\draw [color={rgb, 255:red, 74; green, 144; blue, 226 }  ,draw opacity=1 ]   (348.6,277.8) -- (427.48,160.66) ;
\draw [shift={(428.6,159)}, rotate = 123.96] [color={rgb, 255:red, 74; green, 144; blue, 226 }  ,draw opacity=1 ][line width=0.75]    (10.93,-3.29) .. controls (6.95,-1.4) and (3.31,-0.3) .. (0,0) .. controls (3.31,0.3) and (6.95,1.4) .. (10.93,3.29)   ;
%Shape: Boxed Line [id:dp44738419783976924] 
\draw [color={rgb, 255:red, 208; green, 2; blue, 27 }  ,draw opacity=1 ]   (348.6,277.8) -- (269.49,436.01) ;
\draw [shift={(268.6,437.8)}, rotate = 296.57] [color={rgb, 255:red, 208; green, 2; blue, 27 }  ,draw opacity=1 ][line width=0.75]    (10.93,-3.29) .. controls (6.95,-1.4) and (3.31,-0.3) .. (0,0) .. controls (3.31,0.3) and (6.95,1.4) .. (10.93,3.29)   ;
%Straight Lines [id:da27745396609468553] 
\draw [color={rgb, 255:red, 74; green, 144; blue, 226 }  ,draw opacity=1 ]   (348.6,279) -- (387.71,200.79) ;
\draw [shift={(388.6,199)}, rotate = 116.57] [color={rgb, 255:red, 74; green, 144; blue, 226 }  ,draw opacity=1 ][line width=0.75]    (10.93,-3.29) .. controls (6.95,-1.4) and (3.31,-0.3) .. (0,0) .. controls (3.31,0.3) and (6.95,1.4) .. (10.93,3.29)   ;
%Shape: Boxed Line [id:dp5722244470943598] 
\draw [color={rgb, 255:red, 208; green, 2; blue, 27 }  ,draw opacity=1 ]   (268.6,437.8) -- (190.01,516.39) ;
\draw [shift={(188.6,517.8)}, rotate = 315] [color={rgb, 255:red, 208; green, 2; blue, 27 }  ,draw opacity=1 ][line width=0.75]    (10.93,-3.29) .. controls (6.95,-1.4) and (3.31,-0.3) .. (0,0) .. controls (3.31,0.3) and (6.95,1.4) .. (10.93,3.29)   ;
%Shape: Axis 2D [id:dp9489292928363844] 
\draw  (378.2,277.8) -- (82.2,277.8)(348.6,549.6) -- (348.6,247.6) (89.2,282.8) -- (82.2,277.8) -- (89.2,272.8) (353.6,542.6) -- (348.6,549.6) -- (343.6,542.6)  ;
%Shape: Grid [id:dp458334780755161] 
\draw  [draw opacity=0] (348.6,519) -- (108.6,519) -- (108.6,279) -- (348.6,279) -- cycle ; \draw  [color={rgb, 255:red, 155; green, 155; blue, 155 }  ,draw opacity=0.5 ] (308.6,519) -- (308.6,279)(268.6,519) -- (268.6,279)(228.6,519) -- (228.6,279)(188.6,519) -- (188.6,279)(148.6,519) -- (148.6,279) ; \draw  [color={rgb, 255:red, 155; green, 155; blue, 155 }  ,draw opacity=0.5 ] (348.6,479) -- (108.6,479)(348.6,439) -- (108.6,439)(348.6,399) -- (108.6,399)(348.6,359) -- (108.6,359)(348.6,319) -- (108.6,319) ; \draw  [color={rgb, 255:red, 155; green, 155; blue, 155 }  ,draw opacity=0.5 ] (348.6,519) -- (108.6,519) -- (108.6,279) -- (348.6,279) -- cycle ;

% Text Node
\draw (386.6,199.1) node [anchor=south east] [inner sep=0.75pt]  [color={rgb, 255:red, 74; green, 144; blue, 226 }  ,opacity=1 ]  {$\overline{v}$};
% Text Node
\draw (428.6,155.6) node [anchor=south] [inner sep=0.75pt]  [color={rgb, 255:red, 74; green, 144; blue, 226 }  ,opacity=1 ]  {$\overline{w}$};
% Text Node
\draw (390.6,221.8) node [anchor=north west][inner sep=0.75pt]  [color={rgb, 255:red, 74; green, 144; blue, 226 }  ,opacity=1 ]  {$\overline{v} +\overline{w}$};
% Text Node
\draw (308.6,395.6) node [anchor=south] [inner sep=0.75pt]  [color={rgb, 255:red, 208; green, 2; blue, 27 }  ,opacity=1 ]  {$a\overline{v}$};
% Text Node
\draw (230.6,482.4) node [anchor=north west][inner sep=0.75pt]  [color={rgb, 255:red, 208; green, 2; blue, 27 }  ,opacity=1 ]  {$a\overline{w}$};
% Text Node
\draw (190.6,402.4) node [anchor=north west][inner sep=0.75pt]  [color={rgb, 255:red, 208; green, 2; blue, 27 }  ,opacity=1 ]  {$a\overline{v} +a\overline{w}$};


\end{tikzpicture}

      \caption{$\overline{v} = \langle 1, 2\rangle,\, \overline{w} = \langle 1, 1 \rangle,\, a = -2$}
    \end{figure}

    \setcounter{enumi}{12}

  \item $1\left( -\frac{1}{2} \right) + 2(-1)=-2.5\Rightarrow \left(\frac{-2.5}{\sqrt{5}}\right)\left( \frac{\sqrt{5}}{2} \right)=-1\Rightarrow \cos^{-1}(-1)=0$, thus the angle between them is zero. Because one of the vectors is negative and one is positive, they are in opposite directions. 

  \item $3(-6)+4(-7)=-46\Rightarrow \frac{-46}{(5)(\sqrt{85})} \neq \pm 1$, so they are not parallel

  \item $1(2)+(-2)(-4)+3(5)=25\Rightarrow\frac{25}{(\sqrt{14})(\sqrt{45})}\neq \pm 1$, so they are not parallel

  \item The second vector is a scaled, positive multiple of the first one ($3\overline{v}_1=\overline{v}_2$), so they are parallel and in the same direction

    \setcounter{enumi}{18}

  \item $a = \frac{\sum \overline{F}}{m}=\frac{1}{2}\left( \langle 0, 4 \rangle \right) = \langle 0, 2 \rangle$

  \item $a = \frac{1}{2}\left( \langle -1, 10, 7 \rangle \right) = \langle -.5, 5, 3.5 \rangle$

  \item $a = \frac{1}{2}\left( \bold{i}+\bold{j}+\bold{k} \right) = \frac{1}{2}\bold{i}+\frac{1}{2}\bold{j}+\frac{1}{2}\bold{k}$

    \setcounter{enumi}{22}

  \item $\langle 6, -9 \rangle \Rightarrow \sqrt{36 + 81} = \sqrt{117}$ feet per second

  \item $\langle 7, 4, -2 \rangle \Rightarrow \sqrt{49 + 16 + 4} = \sqrt{69}$ feet per second

    \setcounter{enumi}{26}

  \item $\overline{d} = b - a = \langle -3, -5 \rangle$
    
    \begin{itemize}

      \item Magnitude: $\sqrt{9 + 25} = \sqrt{34}$

      \item Direction: $\frac{1}{\sqrt{34}}\langle -3, -5 \rangle=\langle -\frac{3}{\sqrt{34}}, -\frac{5}{\sqrt{34}} \rangle$

    \end{itemize}

    \setcounter{enumi}{28}

  \item $\overline{d} = \langle -1, -4, -3 \rangle$

    \begin{itemize}

      \item Magnitude: $\sqrt{1 + 16 + 9} = \sqrt{26}$

      \item Direction: $\frac{1}{\sqrt{26}}\langle -1, -4, -3 \rangle = \langle -\frac{1}{\sqrt{26}}, -\frac{4}{\sqrt{26}}, -\frac{3}{\sqrt{26}} \rangle$

     \end{itemize}

    \setcounter{enumi}{32}

  \item $\overline{u} = \frac{\langle 3, 4 \rangle}{5} = \langle \frac{3}{5}, \frac{4}{5} \rangle$

    \begin{itemize}

      \item Magnitude of 3: $3\langle \frac{3}{5}, \frac{4}{5} \rangle = \langle \frac{9}{5}, \frac{12}{5} \rangle$

      \item Magnitude of 7: $7\langle \frac{3}{5}, \frac{4}{5} \rangle = \langle \frac{21}{5}, \frac{28}{5} \rangle$

    \end{itemize}

    \setcounter{enumi}{35}

  \item $\overline{u} = \frac{\langle 2, -1, 3 \rangle}{\sqrt{1 + 4 + 9}} = \langle \frac{2}{\sqrt{14}}, -\frac{1}{\sqrt{14}}, \frac{3}{\sqrt{14}}\rangle$

    \begin{itemize}

      \item Magnitude of 3: $3\langle \frac{2}{\sqrt{14}}, -\frac{1}{\sqrt{14}}, \frac{3}{\sqrt{14}} \rangle = \langle \frac{6}{\sqrt{14}}, -\frac{3}{\sqrt{14}}, \frac{9}{\sqrt{14}} \rangle$

      \item Magnitude of 7: $7\langle \frac{2}{\sqrt{14}}, -\frac{1}{\sqrt{14}}, \frac{3}{\sqrt{14}} \rangle = \langle \frac{14}{\sqrt{14}}, -\frac{7}{\sqrt{14}}, \frac{21}{\sqrt{14}} \rangle$

    \end{itemize}

    \setcounter{enumi}{40}

  \item It becomes -15

  \item It is 13

  \item $\overline{v}=\langle \sqrt{3}, \sqrt{3} \rangle$

    \setcounter{enumi}{44}

  \item $\overline{a}_i = \frac{30\bold{j}}{3} = 10\bold{j} \Rightarrow \overline{g} = -9.81\overline{j} \Rightarrow \sum \overline{a} = 10\bold{j} - 9.81\bold{j} = .19\bold{j}$, so it has an upward acceleration of .19 $\left[ \frac{\si{\meter}}{\si{\second\squared}} \right]$, but the direction of movement can not be determined

  \item $F_g = \frac{Gm_1m_2}{|\overline{r}|^2}\cdot\frac{\overline{r}}{|\overline{r}|} = \frac{6.674(10)^{-11}(3)(5)}{26} \cdot \frac{\langle 1, 3, 4\rangle}{\sqrt{26}} = -G\langle \frac{15}{26\sqrt{26}}, \frac{45}{26\sqrt{26}}, \frac{60}{26\sqrt{26}} \rangle$

\end{enumerate}

\end{document}

