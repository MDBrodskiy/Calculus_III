%%%%%%%%%%%%%%%%%%%%%%%%%%%%%%%%%%%%%%%%%%%%%%%%%%%%%%%%%%%%%%%%%%%%%%%%%%%%%%%%%%%%%%%%%%%%%%%%%%%%%%%%%%%%%%%%%%%%%%%%%%%%%%%%%%%%%%%%%%%%%%%%%%%%%%%%%%%%%%%%%%%
% Written By Michael Brodskiy
% Class: Calculus III (MATH2321)
% Professor: A. Martsinkovsky
%%%%%%%%%%%%%%%%%%%%%%%%%%%%%%%%%%%%%%%%%%%%%%%%%%%%%%%%%%%%%%%%%%%%%%%%%%%%%%%%%%%%%%%%%%%%%%%%%%%%%%%%%%%%%%%%%%%%%%%%%%%%%%%%%%%%%%%%%%%%%%%%%%%%%%%%%%%%%%%%%%%

%%%%%%%%%%%%%%%%%%%%%%%%%%%%%%%%%%%%%%%%%%%%%%%%%%%%%%%%%%%%%%%%%%%%%%%%%%%%%%%%%%%%%%%%%%%%%%%%%%%%%%%%%%%%%%%%%%%%%%%%%%%%%%%%%%%%%%%%%%%%%%%%%%%%%%%%%%%%%%%%%%%
% Written By Michael Brodskiy
% Class: Cornerstone Engineering 1 & 2 (GE1501 & GE1502)
% Professor: B. O'Connell
%%%%%%%%%%%%%%%%%%%%%%%%%%%%%%%%%%%%%%%%%%%%%%%%%%%%%%%%%%%%%%%%%%%%%%%%%%%%%%%%%%%%%%%%%%%%%%%%%%%%%%%%%%%%%%%%%%%%%%%%%%%%%%%%%%%%%%%%%%%%%%%%%%%%%%%%%%%%%%%%%%%

\documentclass[12pt]{article} 
\usepackage{alphalph}
\usepackage[utf8]{inputenc}
\usepackage[russian,english]{babel}
\usepackage{titling}
\usepackage{amsmath}
\usepackage{graphicx}
\usepackage{enumitem}
\usepackage{amssymb}
\usepackage[super]{nth}
\usepackage{everysel}
\usepackage{ragged2e}
\usepackage{geometry}
\usepackage{multicol}
\usepackage{fancyhdr}
\usepackage{cancel}
\usepackage{siunitx}
\geometry{top=1.0in,bottom=1.0in,left=1.0in,right=1.0in}
\newcommand{\subtitle}[1]{%
  \posttitle{%
    \par\end{center}
    \begin{center}\large#1\end{center}
    \vskip0.5em}%

}
\usepackage{hyperref}
\hypersetup{
colorlinks=true,
linkcolor=blue,
filecolor=magenta,      
urlcolor=blue,
citecolor=blue,
}


\title{Homework 1.2}
\date{\today}
\author{Michael Brodskiy\\ \small Professor: A. Martsinkovsky}

\begin{document}

\maketitle

\begin{enumerate}

  \item 

    \begin{itemize}

      \item Magnitude: $\sqrt{3^2 + 4^2}=5$ feet per second

      \item Direction: $\frac{1}{5}\langle 3, 4 \rangle = \langle \frac{3}{5}, \frac{4}{5} \rangle$

    \end{itemize}

    \setcounter{enumi}{2}

  \item

    \begin{itemize}

      \item Magnitude: $\sqrt{(-6)^2 + (1)^2 + (6)^2}=\sqrt{73}$ meters per second

      \item Direction: $\frac{1}{\sqrt{73}}\langle -6, 1, 6 \rangle=\langle \frac{-6}{\sqrt{73}}, \frac{1}{\sqrt{73}}, \frac{6}{\sqrt{73}}\rangle$

    \end{itemize}

    \setcounter{enumi}{4}

  \item

    \begin{itemize}

      \item Magnitude: $\sqrt{(1)^2 + (-1)^2 + (1)^2 + (-1)^2} = 2$

      \item Direction: $\frac{1}{2}\langle 1, -1, 1, -1 \rangle=\langle \frac{1}{2}, -\frac{1}{2}, \frac{1}{2}, -\frac{1}{2}\rangle$

    \end{itemize}

    \setcounter{enumi}{6}

  \item

    \begin{itemize}

      \item Magnitude: $\sqrt{(2)^2 + (-3)^2 + (1)^2}=\sqrt{14}$

      \item Direction: $\frac{1}{14}(2\bold{i}-3\bold{j}+\bold{k}) = \frac{2}{14}\bold{i} - \frac{3}{14}\bold{j} + \frac{1}{14}\bold{k}$

    \end{itemize}

    \setcounter{enumi}{8}

  \item

  \item

    \setcounter{enumi}{12}

  \item $1\left( -\frac{1}{2} \right) + 2(-1)=-2.5\Rightarrow \frac{-2.5}{(\sqrt{5}})\left( \frac{\sqrt{5}}{2} \right)=-1\Rightarrow \cos^{-1}(-1)=0$, thus the angle between them is zero. Because one of the vectors is negative and one is positive, they are in opposite directions. 

  \item

  \item

  \item

    \setcounter{enumi}{18}

  \item

  \item

  \item

    \setcounter{enumi}{22}

  \item

  \item

    \setcounter{enumi}{26}

  \item

    \setcounter{enumi}{28}

  \item

    \setcounter{enumi}{32}

  \item

    \setcounter{enumi}{35}

  \item

    \setcounter{enumi}{40}

  \item

  \item

  \item

    \setcounter{enumi}{44}

  \item

  \item

\end{enumerate}

\end{document}

