%%%%%%%%%%%%%%%%%%%%%%%%%%%%%%%%%%%%%%%%%%%%%%%%%%%%%%%%%%%%%%%%%%%%%%%%%%%%%%%%%%%%%%%%%%%%%%%%%%%%%%%%%%%%%%%%%%%%%%%%%%%%%%%%%%%%%%%%%%%%%%%%%%%%%%%%%%%%%%%%%%%
% Written By Michael Brodskiy
% Class: Calculus III (MATH2321)
% Professor: A. Martsinkovsky
%%%%%%%%%%%%%%%%%%%%%%%%%%%%%%%%%%%%%%%%%%%%%%%%%%%%%%%%%%%%%%%%%%%%%%%%%%%%%%%%%%%%%%%%%%%%%%%%%%%%%%%%%%%%%%%%%%%%%%%%%%%%%%%%%%%%%%%%%%%%%%%%%%%%%%%%%%%%%%%%%%%

%%%%%%%%%%%%%%%%%%%%%%%%%%%%%%%%%%%%%%%%%%%%%%%%%%%%%%%%%%%%%%%%%%%%%%%%%%%%%%%%%%%%%%%%%%%%%%%%%%%%%%%%%%%%%%%%%%%%%%%%%%%%%%%%%%%%%%%%%%%%%%%%%%%%%%%%%%%%%%%%%%%
% Written By Michael Brodskiy
% Class: Cornerstone Engineering 1 & 2 (GE1501 & GE1502)
% Professor: B. O'Connell
%%%%%%%%%%%%%%%%%%%%%%%%%%%%%%%%%%%%%%%%%%%%%%%%%%%%%%%%%%%%%%%%%%%%%%%%%%%%%%%%%%%%%%%%%%%%%%%%%%%%%%%%%%%%%%%%%%%%%%%%%%%%%%%%%%%%%%%%%%%%%%%%%%%%%%%%%%%%%%%%%%%

\documentclass[12pt]{article} 
\usepackage{alphalph}
\usepackage[utf8]{inputenc}
\usepackage[russian,english]{babel}
\usepackage{titling}
\usepackage{amsmath}
\usepackage{graphicx}
\usepackage{enumitem}
\usepackage{amssymb}
\usepackage[super]{nth}
\usepackage{everysel}
\usepackage{ragged2e}
\usepackage{geometry}
\usepackage{multicol}
\usepackage{fancyhdr}
\usepackage{cancel}
\usepackage{siunitx}
\geometry{top=1.0in,bottom=1.0in,left=1.0in,right=1.0in}
\newcommand{\subtitle}[1]{%
  \posttitle{%
    \par\end{center}
    \begin{center}\large#1\end{center}
    \vskip0.5em}%

}
\usepackage{hyperref}
\hypersetup{
colorlinks=true,
linkcolor=blue,
filecolor=magenta,      
urlcolor=blue,
citecolor=blue,
}


\title{Homework 1.2}
\date{\today}
\author{Michael Brodskiy\\ \small Professor: A. Martsinkovsky}

\begin{document}

\maketitle

\begin{enumerate}

  \item 

    \begin{itemize}

      \item Magnitude: $\sqrt{3^2 + 4^2}=5$ feet per second

      \item Direction: $\frac{1}{5}\langle 3, 4 \rangle = \langle \frac{3}{5}, \frac{4}{5} \rangle$

    \end{itemize}

    \setcounter{enumi}{2}

  \item

    \begin{itemize}

      \item Magnitude: $\sqrt{(-6)^2 + (1)^2 + (6)^2}=\sqrt{73}$ meters per second

      \item Direction: $\frac{1}{\sqrt{73}}\langle -6, 1, 6 \rangle=\langle \frac{-6}{\sqrt{73}}, \frac{1}{\sqrt{73}}, \frac{6}{\sqrt{73}}\rangle$

    \end{itemize}

    \setcounter{enumi}{4}

  \item

    \begin{itemize}

      \item Magnitude: $\sqrt{(1)^2 + (-1)^2 + (1)^2 + (-1)^2} = 2$

      \item Direction: $\frac{1}{2}\langle 1, -1, 1, -1 \rangle=\langle \frac{1}{2}, -\frac{1}{2}, \frac{1}{2}, -\frac{1}{2}\rangle$

    \end{itemize}

    \setcounter{enumi}{6}

  \item

    \begin{itemize}

      \item Magnitude: $\sqrt{(2)^2 + (-3)^2 + (1)^2}=\sqrt{14}$

      \item Direction: $\frac{1}{14}(2\bold{i}-3\bold{j}+\bold{k}) = \frac{2}{14}\bold{i} - \frac{3}{14}\bold{j} + \frac{1}{14}\bold{k}$

    \end{itemize}

    \setcounter{enumi}{8}

  \item

  \item

    \setcounter{enumi}{12}

  \item $1\left( -\frac{1}{2} \right) + 2(-1)=-2.5\Rightarrow \left(\frac{-2.5}{\sqrt{5}}\right)\left( \frac{\sqrt{5}}{2} \right)=-1\Rightarrow \cos^{-1}(-1)=0$, thus the angle between them is zero. Because one of the vectors is negative and one is positive, they are in opposite directions. 

  \item $3(-6)+4(-7)=-46\Rightarrow \frac{-46}{(5)(\sqrt{85})} \neq \pm 1$, so they are not parallel

  \item $1(2)+(-2)(-4)+3(5)=25\Rightarrow\frac{25}{(\sqrt{14})(\sqrt{45})}\neq \pm 1$, so they are not parallel

  \item The second vector is a scaled, positive multiple of the first one ($3\overline{v}_1=\overline{v}_2$), so they are parallel and in the same direction

    \setcounter{enumi}{18}

  \item $a = \frac{\sum \overline{F}}{m}=\frac{1}{2}\left( \langle 0, 4 \rangle \right) = \langle 0, 2 \rangle$

  \item $a = \frac{1}{2}\left( \langle -1, 10, 7 \rangle \right) = \langle -.5, 5, 3.5 \rangle$

  \item $a = \frac{1}{2}\left( \bold{i}+\bold{j}+\bold{k} \right) = \frac{1}{2}\bold{i}+\frac{1}{2}\bold{j}+\frac{1}{2}\bold{k}$

    \setcounter{enumi}{22}

  \item $\langle 6, -9 \rangle \Rightarrow \sqrt{36 + 81} = \sqrt{117}$ feet per second

  \item $\langle 7, 4, -2 \rangle \Rightarrow \sqrt{49 + 16 + 4} = \sqrt{69}$ feet per second

    \setcounter{enumi}{26}

  \item $\overline{d} = b - a = \langle -3, -5 \rangle$
    
    \begin{itemize}

      \item Magnitude: $\sqrt{9 + 25} = \sqrt{34}$

      \item Direction: $\frac{1}{\sqrt{34}}\langle -3, -5 \rangle=\langle -\frac{3}{\sqrt{34}}, -\frac{5}{\sqrt{34}} \rangle$

    \end{itemize}

    \setcounter{enumi}{28}

  \item $\overline{d} = \langle -1, -4, -3 \rangle$

    \begin{itemize}

      \item Magnitude: $\sqrt{1 + 16 + 9} = \sqrt{26}$

      \item Direction: $\frac{1}{\sqrt{26}}\langle -1, -4, -3 \rangle = \langle -\frac{1}{\sqrt{26}}, -\frac{4}{\sqrt{26}}, -\frac{3}{\sqrt{26}} \rangle$

     \end{itemize}

    \setcounter{enumi}{32}

  \item $\overline{u} = \frac{\langle 3, 4 \rangle}{5} = \langle \frac{3}{5}, \frac{4}{5} \rangle$

    \begin{itemize}

      \item Magnitude of 3: $3\langle \frac{3}{5}, \frac{4}{5} \rangle = \langle \frac{9}{5}, \frac{12}{5} \rangle$

      \item Magnitude of 7: $7\langle \frac{3}{5}, \frac{4}{5} \rangle = \langle \frac{21}{5}, \frac{28}{5} \rangle$

    \end{itemize}

    \setcounter{enumi}{35}

  \item $\overline{u} = \frac{\langle 2, -1, 3 \rangle}{\sqrt{1 + 4 + 9}} = \langle \frac{2}{\sqrt{14}}, -\frac{1}{\sqrt{14}}, \frac{3}{\sqrt{14}}\rangle$

    \begin{itemize}

      \item Magnitude of 3: $3\langle \frac{2}{\sqrt{14}}, -\frac{1}{\sqrt{14}}, \frac{3}{\sqrt{14}} \rangle = \langle \frac{6}{\sqrt{14}}, -\frac{3}{\sqrt{14}}, \frac{9}{\sqrt{14}} \rangle$

      \item Magnitude of 7: $7\langle \frac{2}{\sqrt{14}}, -\frac{1}{\sqrt{14}}, \frac{3}{\sqrt{14}} \rangle = \langle \frac{14}{\sqrt{14}}, -\frac{7}{\sqrt{14}}, \frac{21}{\sqrt{14}} \rangle$

    \end{itemize}

    \setcounter{enumi}{40}

  \item It becomes -15

  \item It is 13

  \item $\overline{v}=\langle \sqrt{3}, \sqrt{3} \rangle$

    \setcounter{enumi}{44}

  \item $\overline{a}_i = \frac{30\bold{j}}{3} = 10\bold{j} \Rightarrow \overline{g} = -9.81\overline{j} \Rightarrow \sum \overline{a} = 10\bold{j} - 9.81\bold{j} = .19\bold{j}$, so it has an upward acceleration of .19 $\left[ \frac{\si{\meter}}{\si{\second\squared}} \right]$, but the direction of movement can not be determined

  \item $F_g = \frac{Gm_1m_2}{r^2} = \frac{6.674(10)^{-11}(3)(5)}{\langle 1, 3, 4 \rangle^2} = G\langle 15, 135, 240 \rangle$

\end{enumerate}

\end{document}

