%%%%%%%%%%%%%%%%%%%%%%%%%%%%%%%%%%%%%%%%%%%%%%%%%%%%%%%%%%%%%%%%%%%%%%%%%%%%%%%%%%%%%%%%%%%%%%%%%%%%%%%%%%%%%%%%%%%%%%%%%%%%%%%%%%%%%%%%%%%%%%%%%%%%%%%%%%%%%%%%%%%
% Written By Michael Brodskiy
% Class: Calculus III (MATH2321)
% Professor: A. Martsinkovsky
%%%%%%%%%%%%%%%%%%%%%%%%%%%%%%%%%%%%%%%%%%%%%%%%%%%%%%%%%%%%%%%%%%%%%%%%%%%%%%%%%%%%%%%%%%%%%%%%%%%%%%%%%%%%%%%%%%%%%%%%%%%%%%%%%%%%%%%%%%%%%%%%%%%%%%%%%%%%%%%%%%%

%%%%%%%%%%%%%%%%%%%%%%%%%%%%%%%%%%%%%%%%%%%%%%%%%%%%%%%%%%%%%%%%%%%%%%%%%%%%%%%%%%%%%%%%%%%%%%%%%%%%%%%%%%%%%%%%%%%%%%%%%%%%%%%%%%%%%%%%%%%%%%%%%%%%%%%%%%%%%%%%%%%
% Written By Michael Brodskiy
% Class: Calculus III (MATH2321)
% Professor: A. Martsinkovsky
%%%%%%%%%%%%%%%%%%%%%%%%%%%%%%%%%%%%%%%%%%%%%%%%%%%%%%%%%%%%%%%%%%%%%%%%%%%%%%%%%%%%%%%%%%%%%%%%%%%%%%%%%%%%%%%%%%%%%%%%%%%%%%%%%%%%%%%%%%%%%%%%%%%%%%%%%%%%%%%%%%%

\documentclass[12pt]{article} 
\usepackage{alphalph}
\usepackage[utf8]{inputenc}
\usepackage[russian,english]{babel}
\usepackage{titling}
\usepackage{amsmath}
\usepackage{graphicx}
\usepackage{enumitem}
\usepackage{amssymb}
\usepackage[super]{nth}
\usepackage{everysel}
\usepackage{ragged2e}
\usepackage{geometry}
\usepackage{multicol}
\usepackage{fancyhdr}
\usepackage{cancel}
\usepackage{siunitx}
\geometry{top=1.0in,bottom=1.0in,left=1.0in,right=1.0in}
\newcommand{\subtitle}[1]{%
  \posttitle{%
    \par\end{center}
    \begin{center}\large#1\end{center}
    \vskip0.5em}%

}
\usepackage{hyperref}
\hypersetup{
colorlinks=true,
linkcolor=blue,
filecolor=magenta,      
urlcolor=blue,
citecolor=blue,
}


\title{Homework 1.3}
\date{\today}
\author{Michael Brodskiy\\ \small Professor: A. Martsinkovsky}

\begin{document}

\maketitle

\begin{enumerate}

  \item $2(-1) + 3(2) = 4\Rightarrow 4 > 0$, so the angle is acute

  \item $\sqrt{2}(\sqrt{18}) + (3)(-2) = 0$, so the angle is right

  \item $1(3) + 2(-2) + 3(1) = 8 \Rightarrow 8 > 0$, so the angle is acute

  \item $6(-3) + 4(-2) + 2(-1) = -28 \Rightarrow -28 < 0$, so the angle is obtuse

    \setcounter{enumi}{8}

  \item $2(-1) + 3(2) = 4 \Rightarrow \frac{4}{\sqrt{13}\sqrt{5}}\Rightarrow \theta = \cos^{-1}\left( \frac{4}{\sqrt{65}} \right)$

  \item $\sqrt{2}(\sqrt{18}) + 3(-2) = 0 \Rightarrow \theta = \cos^{-1} (0) = \frac{\pi}{2}$

  \item $1(3) + 2(-2) + 3(1) = 8 \Rightarrow \frac{8}{14} \Rightarrow \theta = \cos^{-1} \left( \frac{4}{7} \right)$

  \item $6(-3) + 4(-2) + 2(-1) =-28 \Rightarrow \frac{-28}{\sqrt{56}\sqrt{14}} = -1 \Rightarrow \theta = \cos^{-1}(-1) = \pi$

    \setcounter{enumi}{16}

  \item $2(3) - 3(-1) = 3$

  \item $2(3) - 3(0) = 6$

  \item $0 - 3 + 1 - (-1) = 5$

    \setcounter{enumi}{21}

  \item $\frac{3}{(2)(5)} = .3$

  \item $\sqrt{(4) + 10(0) + 25} = \sqrt{29}$

    \setcounter{enumi}{26}

  \item $(-1)(6) + 3(2) = 0 \Rightarrow \langle -1, 3 \rangle$ is $\perp \langle 6, 2\rangle$ 

  \item $1(\pi) + (-1)(\pi) = 0 \Rightarrow \langle 1, -1 \rangle$ is $\perp \langle \pi, \pi \rangle$

  \item $3(2) + (-2)(3) + 1(0) = 0 \Rightarrow \langle 3, -2, 1 \rangle$ is $\perp \langle 2, 3, 0 \rangle$

  \item None of the vectors are parallel or perpendicular to any of the other ones

    \setcounter{enumi}{32}

  \item $\overline{u}_{\overline{v}} = \frac{\langle 1, -1 \rangle}{\sqrt{2}} = \langle \frac{1}{\sqrt{2}}, -\frac{1}{\sqrt{2}} \rangle \Rightarrow \overline{F}_{\overline{v}} = \left( \frac{2}{\sqrt{2}} + -\frac{3}{\sqrt{2}} \right)\langle \frac{1}{\sqrt{2}}, -\frac{1}{\sqrt{2}} \rangle = \langle -\frac{1}{2}, \frac{1}{2} \rangle \Rightarrow \overline{F}_{\overline{n}} = \langle 2, 3 \rangle - \langle -.5, .5 \rangle = \langle 2.5, 2.5 \rangle$

    \begin{itemize}

      \item $\overline{F}_{\overline{v}} \cdot \overline{F}_{\overline{n}} = (-.5)(2.5) + (.5)(2.5) = 0 \Rightarrow$ Orthogonal \textcolor{green}{\checkmark}

    \end{itemize}

  \item $\overline{u} = \frac{\langle -2, 2 \rangle}{\sqrt{8}} = \langle -\frac{2}{\sqrt{8}}, \frac{2}{\sqrt{8}} \rangle \Rightarrow \overline{F}_{\overline{v}} = \left( -\frac{4}{\sqrt{8}} + \frac{6}{\sqrt{8}} \right)\langle -\frac{2}{\sqrt{8}}, \frac{2}{\sqrt{8}} \rangle = \langle -.5, .5 \rangle\Rightarrow \overline{F}_{\overline{n}} = \langle 2, 3 \rangle - \langle -.5, .5 \rangle = \langle 2.5, 2.5 \rangle$

    \begin{itemize}

      \item $\overline{F}_{\overline{v}} \cdot \overline{F}_{\overline{n}} = (-.5)(2.5) + (.5)(2.5) = 0 \Rightarrow$  Orthogonal \textcolor{green}{\checkmark}

    \end{itemize}

  \item $\overline{u} = \frac{\langle 1, -1, 0 \rangle}{\sqrt{2}} = \langle \frac{1}{\sqrt{2}}, -\frac{1}{\sqrt{2}}, 0 \rangle \Rightarrow \overline{F}_{\overline{v}} = \left(\frac{1}{\sqrt{2}} - \frac{2}{\sqrt{2}} + 0 \right)\langle \frac{1}{\sqrt{2}}, -\frac{1}{\sqrt{2}}, 0 \rangle = \langle -.5, .5, 0 \rangle \Rightarrow \overline{F}_{\overline{n}} = \langle 1, 2, 3 \rangle - \langle -.5, .5, 0 \rangle = \langle 1.5, 1.5, 3 \rangle$

      \begin{itemize}

        \item $\overline{F}_{\overline{v}} \cdot \overline{F}_{\overline{n}} = (-.5)(1.5) + (.5)(1.5) + (3)(0) = 0 \Rightarrow$ Orthogonal \textcolor{green}{\checkmark}

      \end{itemize}

    \setcounter{enumi}{44}

  \item $W = \overline{F}\cdot\overline{d} = (5)(2) + (-1)(3) = 7$ Joules

  \item $W = (2)(-1) + (3)(2) + (1)(-4) = 0$ Joules

  \item $\overline{d} = \langle 1, 7, 6 \rangle \Rightarrow W = (1)(1) + (1)(7) + (1)(6) = 14$ Joules

  \item $\overline{d} = \langle -1, 0, 1 \rangle \Rightarrow W = (2)(-1) + (3)(0) + (1)(1) = -1$ Joule

\end{enumerate}

\end{document}

